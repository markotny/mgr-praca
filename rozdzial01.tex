% !TeX root = ./Dyplom.tex

\chapter{Wstęp}
 
\section{Cel i zakres pracy}
	Celem pracy jest zbadanie skuteczności nowoczesnych metod z dziedziny przetwarzania języka naturalnego na nieotagowanym zestawie danych.
	W ostatnich latach powstały przełomowe rozwiązania oparte na modelach językowych bazujących na architekturze BERT\@.
	W pracy przeprowadzone zostaną badania mające na celu ustalenie, jak sprawują się takie metody dla języka polskiego w kontekście nienadzorowanego problemu modelowania tematycznego.
	Jest to zadanie, w którym model językowy grupuje dokumenty poprzez analizę semantyczną zakładając,
		że zgrupowane dokumenty reprezentują wspólny temat.
	
	Jako źródło dokumentów posłużą wypowiedzi posłów Sejmu III Rzeczpospolitej Polskiej pozyskane z korpusu dyskursu parlamentarnego.
	Z takim zbiorem danych wiążą się dodatkowe wyzwania, mianowicie nie wszystkie wypowiedzi zawierają wyróżniające słowa
		(np.\ kiedy wypowiedź jest odpowiedzią na zapytanie to osoba nie zawsze zawrze w wypowiedzi kontekst).
	Dodatkowo, większość współczesnych metod NLP operuje na krótkich dokumentach, podczas gdy wypowiedzi sejmowe mają formę nierzadko długich przemówień.
	Te aspekty zbioru danych oraz wysokie zróżnicowanie tematyczne kwestii poruszanych w sejmie sprawiają, że jest to intrygujący i obiecujący zestaw dokumentów do analizy.

	Analiza przeprowadzona zostanie na modelach wykorzystujących trzy alternatywne podejścia kodowania dokumentów:
	\begin{itemize}
		\item architektura BERT,
		\item sieć konwolucyjna,
		\item TF-IDF (metoda statystyczna).
	\end{itemize}
	Ponadto, testy wykonane zostaną dla różnych kombinacji parametrów wykorzystywanych algorytmów do redukcji wymiarowości i grupowania wektorów,
		aby odnaleźć optymalne wartości dla analizowanego korpusu.

	Jako alternatywną metodę modelowania tematycznego dokumentów, która stanowi punkt odniesienia,
		przeprowadzona zostanie analiza modelem LDA\@.
	Jest to obecnie jedno z najpopularniejszych narzędzi przeprowadzania modelowania tematycznego\cite{LDA_popularity},
		które bazuje na metodach statystycznych.

\section{Układ pracy}
	W kolejnym rozdziale omówiony zostanie wykorzystywany korpus wypowiedzi sejmowych,
		oraz działania jakie zostały wykonane podczas jego przetwarzania.
	W rozdziale trzecim przedstawiono wszystkie analizowane metody generowania wektorów dokumentów.
	Rozdział czwarty zawiera opisy kolejnych kroków wykonywanych w ramach modelowania tematycznego
		--- redukcja wymiarowości wektorów, grupowanie oraz generowanie reprezentacji tematu.
	W kolejnym rozdziale przedstawione zostały sposoby ewaluacji tematów odnalezionych w korpusie.
	W szóstym rozdziale opisano uzyskane wyniki zarówno pod kątem wykorzystywanych metod i parametrów, jak i zastosowanych metryk.
	W ostatnim rozdziale zawarto podsumowanie pracy.