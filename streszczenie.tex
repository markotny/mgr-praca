\pdfbookmark[0]{Streszczenie}{streszczenie.1}
%\phantomsection
%\addcontentsline{toc}{chapter}{Streszczenie}
%%% Poniższe zostało niewykorzystane (tj. zrezygnowano z utworzenia nienumerowanego rozdziału na abstrakt)
%%%\begingroup
%%%\setlength\beforechapskip{48pt} % z jakiegoś powodu była maleńka różnica w położeniu nagłówka rozdziału numerowanego i nienumerowanego
%%%\chapter*{\centering Abstrakt}
%%%\endgroup
%%%\label{sec:abstrakt}
%%%Lorem ipsum dolor sit amet eleifend et, congue arcu. Morbi tellus sit amet, massa. Vivamus est id risus. Sed sit amet, libero. Aenean ac ipsum. Mauris vel lectus. 
%%%
%%%Nam id nulla a adipiscing tortor, dictum ut, lobortis urna. Donec non dui. Cras tempus orci ipsum, molestie quis, lacinia varius nunc, rhoncus purus, consectetuer congue risus. 
%\mbox{}\vspace{2cm} % można przesunąć, w zależności od długości streszczenia
\begin{abstract}
	W ostatnich latach powstały przełomowe rozwiązania z dziedziny przetwarzania języka, oparte na modelach językowych bazujących na Transformerach.
	W pracy zbadano skuteczność takich modeli dla języka polskiego w kontekście problemu modelowania tematycznego.
	Algorytmy porównano z wiodącą metodą LDA\@.
	Testy przeprowadzono na Korpusie Dyskursu Parlamentarnego, który zawiera różnorodne i długie wypowiedzi.
	Modelowanie tematyczne oparto na algorytmie BERTopic, który składa się z następujących kroków:
		osadzenie wektorów zdań, redukcja wymiarowości wektorów i klastrowanie w grupy tematyczne poprzez ocenę podobieństwa semantycznego.
	
		Z powodzeniem utworzono modele przewyższające jakością LDA, choć nie wszystkie pierwotne założenia okazały się słuszne.
	Jak pokazała analiza wyników, dokładność zaawansowanych modeli takich jak \emph{Sentence-Transformers} czy \emph{Universal Sentence Encoder}
		maleje dla długich wypowiedzi.
	Okazuje się, że analizując tak długie teksty, lepiej jest skorzystać ze zwykłych metod statystycznych \emph{TF-IDF} poprzedzonych lematyzacją danych wejściowych.
	Wektory wygenerowane w taki sposób efektywniej wykorzystują duży rozmiar dokumentów,
		podczas gdy pozostałe modele osiągają gorszą dokładność wynikającą z uśredniania wektorów zdań bądź analizy tylko początkowych fragmentów wypowiedzi.
\end{abstract}
\mykeywords{nlp, bert, bertopic, lda, modelowanie tematyczne}\\ 
% Dobrze byłoby skopiować słowa kluczowe do metadanych dokumentu pdf (w pliku Dyplom.tex)
% Niestety, zaimplementowane makro nie robi tego z automatu, więc pozostaje kopiowanie ręczne.

{
\selectlanguage{english}
\begin{abstract}
	Recent advancements in Natural Language Processing produced state of the art language models based on Transformers.
	This study aims to determine, how well these models perform for topic modeling task in polish language.
	With LDA being the most commonly used method, it is assumed as the baseline.
	The tests have been carried out on the Polish Parliamentary Corpus, which consists of diverse and long utterances.
	Topic modeling with is based on BERTopic, which consists of the following steps:
		sentence embedding, vector dimensionality reduction and clustering into topic groups by comparing semantic similarity.
	
		It has shown promising results far exceeding those achieved by LDA,
		however embedding lemmatized sentences with statistical \emph{TF-IDF} method has proven to be most efficient for such long sequences.
	Other methods (\emph{Sentence-Transformers}, \emph{Universal Sentence Encoder}) lose their accuracy with document length increase.
	This is caused by limited input length when analysing whole utterance, or information loss in averaging vectors when splitting document into sentences.
\end{abstract}
\mykeywords{nlp, bert, bertopic, lda, topic modeling}
}
