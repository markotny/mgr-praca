% !TeX root = ./Dyplom.tex

\chapter{Korpus}
  W pracy skorzystano z Korpusu Dyskursu Parlamentarnego\cite{PPC}, zwanym dalej PPC.
  Korpus zawiera dane sejmowe wygenerowane z zapisów stenograficznych z lat 1991-2019
    oraz ze zdigitalizowanych transkrypcji z lat 1918-1990.
  Zawarte są również dane z senatu oraz posiedzeń komisji i interpelacji,
    jednak w tej pracy skupiono się na danych z posiedzeń plenarnych sejmu III RP.
  Są to posiedzenia od początku Sejmu I Kadencji (25 listopada 1991), wybranego w pierwszych po wojnie w pełni wolnych wyborach.

\section{Format danych}
  Korpus dostępny jest w formacie bazującym na XML TEI P5, opracowanym na potrzeby NKJP \textit{(Narodowy Korpus Języka Polskiego)}.
  Każde posiedzenie udokumentowane jest szeregiem plików reprezentujących inne aspekty analizy tekstu.
  Dostępne są m.in. znaczniki morfosyntaktyczne, lematy, czy grupy syntaktyczne.
  Pełna struktura opisana została w \cite[Ogrodniczuk, 2012]{PSC},
    natomiast na potrzeby tej pracy przetworzone zostały jedynie następujące pliki:
  \begin{itemize}
    \item \verb|header.xml| --- metadane posiedzenia (data, lista mówców),
    \item \verb|text_structure.xml| --- kolejne wypowiedzi z oznaczonym mówcą.
  \end{itemize}
  Adnotacje lingwistyczne nie zostały uwzględnione,
    gdyż ze względu na specyfikę poruszanego problemu wymagane są surowe, nieoznakowane teksty.
  
\section{Przetwarzanie danych}
  W celu ograniczenia liczby wypowiedzi, które nie są istotne z punktu widzenia analizy,
    wykonano szereg czynności mających na celu odfiltrowanie nierzeczowych fragmentów.
  
  Zdarza się, że w trakcie wypowiedzi występują wtrącenia innych osób.
  Takie komentarze są jednak odpowiednio oznakowane (,,\#komentarz'', czy ,,\#głosZSali'').
  Dodatkowo, wszelkie komentarze do wypowiedzi występują jako osobny wpis, którego treść zawarta jest w nawiasach.
  Takie fragmenty są usuwane, aby nie zaburzyć treści wypowiedzi danej osoby.

  Ponadto przyjęto, że wypowiedzi marszałków oraz wicemarszałków ograniczają się do zarządzania procedowaniem sejmu,
    więc nie wprowadzają żadnej wartości.
  Na końcu usunięto wypowiedzi, których liczba znaków nie przekracza dwustu.
  Próg ten wyznaczony został empirycznie tak, aby odrzucić wypowiedzi nie zawierające żadnej konkretnej informacji.

\section{Uzupełnienie danych o posłach}
  Korpus uzupełniono o dane na temat osób wypowiadających się.
  W tym celu skorzystano z internetowego Archiwum Danych o Posłach\cite{archiwum_posl}.
  Na stronie dostępne są dane z różnym poziomem szczegółowości o posłach wszystkich kadencji od roku 1952.
  Z archiwum pobrano następujące informacje dla wszystkich posłów III RP: klub parlamentarny, lista wyborcza, partia oraz okręg wyborczy.

  Dane o posłach powiązano ze wpisami w korpusie na podstawie imienia i nazwiska mówcy.
  W ten sposób udało się dopasować zdecydowaną większość wypowiadających się --- ponad 91\%.
  W pozostałych przypadkach wypowiadają się zwykle osoby nie z rządu, np.\ prezydent.

  Dokonano ręcznej weryfikacji spójności danych o okręgach wyborczych i uwspólniono okręgi z większych miast.
  Następnie do każdego okręgu dopasowano województwo.

\section{Końcowy format korpusu}
  Po przetworzeniu korpusu otrzymano 291415 wypowiedzi na przestrzeni 28 lat.
  Dla każdej z nich dostępne są następujące informacje:
  \begin{itemize}
    \item id --- identyfikator wypowiedzi, który składa się z identyfikatora posiedzenia (z PPC) połączonego z tagiem wypowiedzi (np. \verb|div-123|),
    \item mówca --- imię i nazwisko tak, jak występuje bezpośrednio w PPC,
    \item data --- data posiedzenia, z którego pochodzi dana wypowiedź,
    \item text --- treść wypowiedzi,
    \item poseł* --- imię i nazwisko posła wzięte z archiwum,
    \item klub* --- klub parlamentarny, do którego należy poseł,
    \item lista* --- lista wyborcza, z której startował poseł,
    \item partia** --- partia, do której należy poseł,
    \item okręg* --- okręg wyborczy posła,
    \item kadencja --- kadencja, z której pochodzi dana wypowiedź
  \end{itemize}
  \vspace{1em}
  * dostępne dla ok.\ 91\% wypowiedzi\\
  ** dostępne dla ok.\ 18\% wypowiedzi