% !TeX root = ./Dyplom.tex

\chapter{Analiza skuteczności modeli}
\todo{Opisać\cite{Eval_Topics}\cite{U_MASS}\cite{C_NPMI}}

\section{Ewaluacja spójności}
	Ze względu na specyfikę korpusu możliwe jest również sformułowanie metryki wykorzystującej porządek obrad sejmu.
	Każde posiedzenie przebiega zgodnie z ustaloną strukturą, gdzie marszałek zarządza obrady na temat danej ustawy, poprawki czy interpelacji,
		a następnie kolejni posłowie wyrażają swoją opinię na ten temat.
	Można więc założyć, że w obrębie debaty nad jednym punktem obrad, wypowiedzi powinny zostać zakwalifikowane do tego samego zbioru tematycznego.
	Stworzono więc algorytm wykorzystujący tę zależność, przedstawiony na listingu\ref{lst:consistency}.
	Założono, że w zdecydowanej większości przypadków, kolejne dwie wypowiedzi powinny mieć przypisany ten sam temat.

	Algorytm analizuje osobno każdy dzień posiedzeń.
	Wypowiedzi sortowane są zgodnie z kolejnością wynikającą z transkrypcji przemówień.
	Jeśli w ciągu dnia jest mniej niż dwie wypowiedzi, to dzień jest pomijany.
	W zbiorze danych występuje jeden taki przypadek, natomiast mediana wynosi 138.
	Wśród wypowiedzi danego dnia porównywane są tematy par sąsiadujących wypowiedzi.
	Zliczane są zarówno znalezione pasujące pary, jak i różniące się.
	Jeśli temat analizowanej wypowiedzi jest nieznany (algorytm HDBSCAN nie przypisał dokumentu do żadnej grupy),
		to jest ona pomijana.
	Mechanizm ten zapobiega sytuacji, w której duża liczba nieprzypisanych dokumentów zawyżałaby wynik.
	Dzieje się tak ze względu na to, że nieprzypisane dokumenty często stanowią znaczną część zbioru (30--50\%),
		przez co istnieje duże prawdopodobieństwo znalezienia par takich dokumentów.
	Dodatkowo od sumy różniących się par odejmowana jest liczba o 1 mniejsza od liczby tematów danego dnia.
	Działanie to rekompensuje przejścia między tematami, które muszą wystąpić i nie są błędem.
	Takich przejść w ciągu dnia wystąpi co najmniej tyle, ile jest tematów minus jeden.
	Końcowy wynik obliczany jest jako stosunek zgadzających się par do wszystkich zliczonych par.
	
	\begin{lstlisting}[style=algorithm,label=lst:consistency,caption=Algorytm obliczania spójności tematów]
input: speeches sorted by original order of speaking with date and assigned topic identifier
output: consistency score between 0 and 1 with higher score indicating more consistent topics
	
same_per_day := []
different_per_day := []

for each speeches in data grouped by date do
	if len(speeches) < 2
		continue
	
	same := 0
	different := 0
	for i in range(len(speeches) - 1) do
		if speeches[i].topic is unknown
			continue

		if speeches[i].topic = speeches[i + 1].topic
			same := same + 1
		else
			different := different + 1

	topic_count := len(speeches.topic.unique())
	different := different - (topic_count - 1)
	
	same_per_day.append(same)
	different_per_day.append(different)

score = sum(same_per_day) / sum(same_per_day, different_per_day)
return score
	\end{lstlisting}