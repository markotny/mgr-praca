\chapter{Redakcja pracy}
\section{Układ pracy}
Istnieje wiele sposobów organizacji poszczególnych rozdziałów pracy. W dokumentacji klasy memoir (\url{http://tug.ctan.org/tex-archive/macros/latex/contrib/memoir/memman.pdf}) można np.\ znaleźć zalecenia co do edytorskiej strony amerykańskich prac dyplomowych. Na stronie 363 można przeczytać, że część początkowa pracy może zawierać:
\begin{enumerate}
\item Title page
\item Approval page
\item Abstract
\item Dedication (optional)
\item Acknowledgements (optional)
\item Table of contents
\item List of tables (if there are any tables)
\item List of figures (if there are any figures)
\item Other lists (e.g., nomenclature, definitions, glossary of terms, etc.)
\item Preface (optional but must be less than ten pages)
\item Under special circumstances further sections may be allowed
\end{enumerate}
A dalej kolejne wymagania, w tym informacje o zawartości części końcowej:
\begin{enumerate}
\item Notes (if you are using endnotes and grouping them at the end)
\item References (AKA ‘Bibliography’ or ‘Works Cited’)
\item Appendices
\item Biographical sketch (optional)
\end{enumerate}

Prace dyplomowe powstające na Wydziale Elektroniki Politechniki Wrocławskiej standardowo redaguje się w następującym układzie:
\noindent\fbox{\begin{minipage}{\dimexpr\textwidth-2\fboxsep-2\fboxrule\relax}
\begin{quote}
\item Strona tytułowa
\item Streszczenie 
\item Strona z dedykacją (opcjonalna)
\item Spis treści  
\item Spis rysunków (opcjonalny)
\item Spis tabel (opcjonalny)
\item Spis listingów (opcjonalny)
\item Skróty (wykaz opcjonalny)
\item Abstrakt 
\item 1. Wstęp 
\begin{quote}
\item 1.1 Wprowadzenie 
\item 1.2 Cel i zakres pracy 
\item 1.3 Układ pracy 
\end{quote}
\item 2. Kolejny rozdział
\begin{quote}
\item 2.1 Sekcja
\begin{quote}
\item 2.1.1 Podsekcja
\begin{quote}
\item Nienumerowana podpodsekcja
\begin{quote}
\item Paragraf
\end{quote}
\end{quote}
\end{quote}
\end{quote}
\item $\ldots$
\item \#. Podsumownie i wnioski
\item Literatura
\item A. Dodatek
\begin{quote}
\item A.1 Sekcja w dodatku
\end{quote}
\item $\ldots$
\item \$. Instrukcja wdrożeniowa
\item \$. Zawartość płyty CD/DVD
\item Indeks rzeczowy (opcjonalny)
\end{quote}
\end{minipage}}\\

Streszczenie -- syntetyczny opis tego, o czym jest i co zawiera praca, zwykle jednostronicowy, po polsku i po angielsku (podczas wysyłania pracy do analizy antyplagiatowej należy skopiować tekst z tej sekcji do formularza w systemie ASAP).

Spis treści -- powinien być generowany automatycznie, z podaniem tytułów i numerów stron. Typ czcionki oraz wielkość liter spisu treści powinny być takie same jak w niniejszym wzorcu.

Spis rysunków, Spis tabel -- powinny być generowane automatycznie (podobnie jak Spis treści). Elementy te są opcjonalne (robienie osobnego spisu, w którym na przykład są tylko dwie pozycje specjalnie nie ma sensu).

Skróty -- jeśli w pracy występują liczne skróty, to w tej części należy podać ich rozwinięcia.

Wstęp -- pierwszy rozdział, w którym powinien znaleźć się opis dziedziny, w jakiej osadzona jest praca, oraz wyjaśnienie motywacji do podjęcia tematu (na to przeznaczona jest sekcja ,,Wprowadzenie''). W sekcji ,,Cel i zakres'' powinien znaleźć się opis celu oraz zadań do wykonania, zaś w sekcji ,,Układ pracy'' -- opis zawartości kolejnych rozdziałów.

Podsumowanie -- w rozdziale tym powinny być zamieszczone: podsumowanie uzyskanych efektów oraz wnioski końcowe wynikające z realizacji celu pracy dyplomowej.

Literatura -- wykaz źródeł wykorzystanych w pracy (do każdego źródła musi istnieć odpowiednie cytowanie w tekście). Wykaz ten powinien być generowany automatycznie.

Dodatki -- miejsce na zamieszczanie informacji dodatkowych, jak: Instrukcja wdrożeniowa, Instrukcja uruchomieniowa, Podręcznik użytkownika itp.
Osobny dodatek powinien być przeznaczony na opis zawartości dołączonej płyty CD/DVD. Założono, że będzie to zawsze ostatni dodatek.

Indeks rzeczowy -- miejsce na zamieszczenie kluczowych wyrazów, do których czytelnik będzie chciał sięgnąć. Indeks powinien być generowany automatycznie. Jego załączanie jest opcjonalne.
\section{Styl}
\label{sec:Styl}
Zasady pisania pracy (przy okazji można tu zaobserwować efekt wyrównania wpisów występujących na liście wyliczeniowej uzależnione od długości etykiety):
\begin{enumerate}[labelwidth=\widthof{\ref{last-item}},label=\arabic*.]
\item Praca dyplomowa powinna być napisana w  formie bezosobowej (,,w pracy pokazano ...''). Taki styl przyjęto na uczelniach w naszym kraju, choć w krajach anglosaskich preferuje się redagowanie treści w pierwszej osobie.
\item W tekście pracy można odwołać się do myśli autora, ale nie w pierwszej osobie, tylko poprzez wyrażenia typu: ,,autor wykazał, że ...''. 
\item Odwołując się do rysunków i tabel należy używać zwrotów typu: ,,na rysunku pokazano ...'', ,,w tabeli zamieszczono ...'' (tabela i rysunek to twory nieżywotne, więc ,,rysunek pokazuje'' jest niepoprawnym zwrotem).
\item Praca powinna być napisana językiem formalnym, bez wyrażeń żargonowych (,,sejwowanie'' i ,,downloadowanie''), nieformalnych czy zbyt ozdobnych (,,najznamienitszym przykładem tego niebywałego postępu ...'')
\item Pisząc pracę należy dbać o poprawność stylistyczną wypowiedzi
\begin{itemize}
\item trzeba pamiętać, do czego stosuje się ,,liczba'', a do czego ,,ilość'',
\item nie ,,szereg funkcji'' tylko ,,wiele funkcji'',
\item redagowane zdania nie powinny być zbyt długie (lepiej podzielić zdanie wielokrotnie złożone na pojedyncze zdania),
\item itp.
\end{itemize}
\item Zawartość rozdziałów powinna być dobrze wyważona. Nie wolno więc generować sekcji i podsekcji, które mają zbyt mało tekstu lub znacząco różnią się objętością. Zbyt krótkie podrozdziały można zaobserwować w przykładowym rozdziale~\ref{chap:podsumowanie}.
\item Niedopuszczalne jest pozostawienie w pracy błędów ortograficznych czy tzw.\ literówek -- można je przecież znaleźć i skorygować
automatycznie. \addtocounter{enumi}{9997} 
\item  Niedopuszczalne jest pozostawienie w pracy błędów ortograficznych czy tzw.\ literówek -- można je przecież znaleźć i skorygować
automatycznie. \label{last-item}
\end{enumerate}