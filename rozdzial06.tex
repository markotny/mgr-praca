% !TeX root = ./Dyplom.tex

\chapter{Analiza wyników}
	Szczegółowe wyniki zostały przedstawione w rozdziale~\ref{sec:all_scores}.
	W tabeli zawarte są wszystkie metody, dla których wykonane zostały pełne testy (jak opisano w rozdziale~\ref{sec:test_plan}).
	W kolumnach przedstawiono kolejno parametry testów, liczbę znalezionych tematów i liczbę nieprzypisanych dokumentów, wyniki testów oraz wynik uśredniony.
	Jeśli dla danych parametrów liczba tematów była poza zakresem [50--1000], to niektóre testy były pomijane w celach optymalizacyjnych.
	Wtedy w kolumnach wpisana jest wartość ,,<NA>''.

	Wykres przedstawiony na rysunku~\ref{fig:avg_scores} ilustruje końcowe uśrednione wyniki.
	W kolejnych rozdziałach omówione są te wyniki z podziałem na poszczególne modele.

\section{SBERT}
	Wektory wygenerowane przez modele SBERT oparte na transformerach są w stanie generować bardzo dobre wyniki,
		ale wymagana jest analiza poszczególnych zdań, a nie całych wypowiedzi.
	Jak pokazują wyniki uzyskane przez model \emph{sbert-default} (wiersze 0--83 tabeli w rozdziale~\ref{sec:all_scores}),
		w zdecydowanej większości kombinacji parametrów otrzymane grupowania były bardzo ogólne --- od 5 do 25 tematów zwróciło aż 65\% testowanych wariantów.
	Spowodowane jest to dostrojeniem modeli SBERT na korpusach złożonych ze zdań.
	Model osiąga więc najwyższą dokładność analizując zdania standardowej długości, podczas gdy wypowiedzi sejmowe są znacznie dłuższe (rys.~\ref{fig:word_count}).
	Analizując więc tak długie dokumenty, model bierze pod uwagę tylko początki wypowiedzi.
	Te z kolei zwykle rozpoczynają się od formalnego przywitania, np. ,,Szanowna Pani Marszałek!'', ,,Szanowni Państwo!'' czy ,,Wysoka Izbo!''.
	Stąd model, który analizuje całość wypowiedzi począwszy od przywitań, często podda analizie jedynie niewielki fragment merytorycznej części wypowiedzi.
	Efekt ten jest szczególnie uwypuklony w przypadku modeli SBERT, które wyuczone są na zdaniach krótszych niż maksymalny limit wynikający z architektury
		(który dla modeli bazujących na BERT wynosi 512 tokenów).
	Tematy wygenerowane przez model o indeksie w tabeli z wynikami \textbf{37} przedstawione zostały w tabeli~\ref{tab:sbert_default_topics}.
	Można podejrzewać, że temat nr.~\emph{A} zawiera wypowiedzi rozpoczynające się od komentarza kierowanego do innego posła, a w temacie~\emph{C} przemawiający odnoszą się do agendy danego dnia.
	Żadne z tych tematów nie nadają się do dalszej analizy.

	\begin{table}[htb]
		\caption{Reprezentacje wybranych tematów dla modelu sbert-default z parametrami (15,100,100)}\label{tab:sbert_default_topics} % chktex 9 chktex 10
		\centering
		\small
		\begin{tabularx}{\textwidth}{rl}
			\toprule
			A & jerzy | stanisław | kazimierz | henryk | płażyński \\
			B & ustawy | państwa | komisji | europejskiej | polski \\
			C & godz 12 | wspólnie komisją | godz 13 | godz 11 | godz 16 \\
			D & wspólnie komisją | godz 11 | godz 12 | następujących komisji | posiedzenia następujących komisji \\
			\bottomrule
		\end{tabularx}
	\end{table}

	Warto jednak zauważyć, iż nawet ten model po dostrojeniu parametrów potrafi wygenerować czytelne tematy.
	Oznacza to, że część merytorycznych informacji została przeanalizowana podczas generowania wektorów.
	Są one wydobywane, jeśli parametry UMAP oraz HDBSCAN zostaną dostrojone tak,
		aby brane pod uwagę były tylko lokalne zależności (niskie \verb|n_neighbors| oraz \verb|min_samples|).
	Wraz ze wzrostem wartości tych parametrów, brane są pod uwagę ogólniejsze podobieństwa,
		do w przypadku takiego modelu oznacza formalne przywitania, czy nazwiska posłów.
	Jednak nawet najlepszy model (indeks \textbf{34}) jako szum oznaczył prawie 165 tysięcy wypowiedzi, podczas gdy mediana wynosi ok.~137 tysięcy.
	Wynika to z tego, że nawet jeśli model był w stanie wyciągnąć informacje z początku wypowiedzi, to nie zawsze było to wystarczające, przez co więcej dokumentów zostało zaklasyfikowanych jako szum.
	Średni wynik tego modelu wynoszący w zaokrągleniu \(0.45\) również jest znacznie poniżej mediany \(0.57\).

	Zdecydowanie lepiej sprawdziły się warianty modelu, które w pierwszej kolejności dzieliły wypowiedź na zdania, a następnie łączyły wektory wygenerowane dla każdego zdania osobno.
	Najlepszy wynik osiągnął model ważący zdania algorytmem LexRank (10 najwyższych wyników przedstawiono w tabeli~\ref{tab:sbert_top}).
	Z trzech analizowanych metod podziału na zdania, metoda oparta o ważenie wektorami tf-idf wypadła najgorzej, co sugeruje iż metoda ta powoduje zbyt dużą utratę informacji.
	Na podstawie wyników nie można również jednoznacznie stwierdzić, iż ważenie wektorów algorytmem LexRank znacząco podnosi jakość wektorów.
	Co prawda najlepszy wynik osiągnął algorytm ważony tym sposobem,
		jednak mediana wyników najwyższa jest dla algorytmu \emph{sbert-mean} (0.592 vs 0.579 dla \emph{sbert-lexrank-weighted} oraz 0.565 dla \emph{sbert-tf-idf-weighted}).
	Wyniki te skłaniają ku wnioskom, że próby ważenia wektorów skutkują utratą informacji, a nie wytłumianiem szumu, jak pierwotnie zakładano.
	Jak pokazują wyniki na wykresie~\ref{fig:consistency_score}, wektory ważone algorytmem tf-idf osiągają znacznie gorsze wyniki,
		podczas gdy uśrednione i ważone algorytmem LexRank są do siebie zbliżone.
	W pozostałych metrykach wszystkie trzy metody osiągały podobne wyniki co oznacza, że ważenie wektorów w najlepszym wypadku jedynie nie pogarsza wyniku.
	
	W przypadku modeli \emph{top1} traktujących wektor o najwyższym wyniku (lub średnią pięciu wektorów --- \emph{top5}) wg.\ rankingu LexRank lub tfidf,
		wyniki były znacznie gorsze od pozostałych ze względu na utratę zbyt dużej ilości danych.
	Podobnych przyczyn (aczkolwiek w mniejszym stopniu) można się doszukiwać w przyczynach gorszej dokładności modeli ważonych.
	Okazuje się, że każda informacja zawarta w wektorach zdań może okazać się istotna i podczas analizy wielkiej ilości danych,
		informacje te mogą okazać się istotne.
	Z tego powodu najlepsze wyniki osiągnął model \emph{sbert-mean}, który uśrednia wektory wszystkich zdań bez wag.

	\begin{table}[htb]
		\caption{Wyniki 10 najlepszych modeli SBERT}\label{tab:sbert_top}
		\centering
		\begin{tabular}{llrrrr}
			\toprule
			{}  &	\emph{model} &  \emph{n\_neighbors} &  \emph{min\_cluster\_size} &  \emph{min\_samples} &   \emph{avg} \\
			\midrule
			118 &  sbert-lexrank-weighted &           15 &               100 &            1 &  0.679364 \\
			202 &              sbert-mean &           15 &               100 &            1 &  0.658695 \\
			92  &  sbert-lexrank-weighted &            5 &               100 &           50 &  0.653523 \\
			201 &              sbert-mean &           15 &                50 &           50 &  0.647214 \\
			285 &   sbert-tf-idf-weighted &           15 &                50 &           50 &  0.642425 \\
			230 &              sbert-mean &           25 &               100 &            1 &  0.640095 \\
			107 &  sbert-lexrank-weighted &           10 &               100 &          100 &  0.638268 \\
			187 &              sbert-mean &           10 &                50 &           50 &  0.636077 \\
			191 &              sbert-mean &           10 &               100 &          100 &  0.635758 \\
			132 &  sbert-lexrank-weighted &           20 &               100 &            1 &  0.632976 \\
			\bottomrule
			\end{tabular}
	\end{table}
