% !TeX root = ./Dyplom.tex

\chapter{Analiza wyników}
	Szczegółowe wyniki zostały przedstawione w rozdziale~\ref{sec:all_scores}.
	W tabeli zawarte są wszystkie metody, dla których wykonane zostały pełne testy (jak opisano w rozdziale~\ref{sec:test_plan}).
	W kolumnach przedstawiono kolejno parametry testów, liczbę znalezionych tematów i liczbę nieprzypisanych dokumentów, wyniki testów oraz wynik uśredniony.
	Jeśli dla danych parametrów liczba tematów była poza zakresem [50--1000], to niektóre testy były pomijane w celach optymalizacyjnych.
	Wtedy w kolumnach wpisana jest wartość ,,<NA>''.

	Wykres przedstawiony na rysunku~\ref{fig:avg_scores} ilustruje końcowe uśrednione wyniki.
	W kolejnych rozdziałach omówione są te wyniki z podziałem na poszczególne modele.

\section{Porównanie metod generowania wektorów}
	W pierwszej kolejności porównane zostaną wyniki pod kątem metod utworzenia wektorów wypowiedzi.
	Wyniki omawiane są zgodnie z kolejnością występowania w tabeli z wynikami zamieszczonej w rozdziale~\ref{sec:all_scores}.

	\subsection{Sentence Transformers}\label{sec:sbert_summary}
		Wektory wygenerowane przez modele SBERT oparte na transformerach są w stanie generować bardzo dobre wyniki,
			ale wymagana jest analiza poszczególnych zdań, a nie całych wypowiedzi.
		Jak pokazują wyniki uzyskane przez model \emph{sbert-default} (wiersze 0--83 tabeli z wynikami),
			w zdecydowanej większości kombinacji parametrów otrzymane grupowania były bardzo ogólne --- od 5 do 25 tematów zwróciło aż 65\% testowanych wariantów.
		Spowodowane jest to dostrojeniem modeli SBERT na korpusach złożonych ze zdań.
		Model osiąga więc najwyższą dokładność analizując zdania standardowej długości, podczas gdy wypowiedzi sejmowe są znacznie dłuższe (rys.~\ref{fig:word_count}).
		Analizując więc tak długie dokumenty, model bierze pod uwagę tylko początki wypowiedzi.
		Te z kolei zwykle rozpoczynają się od formalnego przywitania, np. ,,Szanowna Pani Marszałek!'', ,,Szanowni Państwo!'' czy ,,Wysoka Izbo!''.
		Stąd model, który analizuje całość wypowiedzi począwszy od przywitań, często podda analizie jedynie niewielki fragment merytorycznej części wypowiedzi.
		Efekt ten jest szczególnie uwypuklony w przypadku modeli SBERT, które wyuczone są na zdaniach krótszych niż maksymalny limit wynikający z architektury
			(który dla modeli bazujących na BERT wynosi 512 tokenów).
		Tematy wygenerowane przez model o indeksie w tabeli z wynikami \textbf{37} przedstawione zostały w tabeli~\ref{tab:sbert_default_topics}.
		Można podejrzewać, że temat nr.~\emph{A} zawiera wypowiedzi rozpoczynające się od komentarza kierowanego do innego posła, a w temacie~\emph{C} przemawiający odnoszą się do agendy danego dnia.
		Żadne z tych tematów nie nadają się do dalszej analizy.

		\begin{table}[htb]
			\caption{Reprezentacje wybranych tematów dla modelu sbert-default z parametrami (15,100,100)}\label{tab:sbert_default_topics} % chktex 9 chktex 10
			\centering
			\small
			\begin{tabularx}{\textwidth}{rl}
				\toprule
				A & jerzy | stanisław | kazimierz | henryk | płażyński \\
				B & ustawy | państwa | komisji | europejskiej | polski \\
				C & godz 12 | wspólnie komisją | godz 13 | godz 11 | godz 16 \\
				D & wspólnie komisją | godz 11 | godz 12 | następujących komisji | posiedzenia następujących komisji \\
				\bottomrule
			\end{tabularx}
		\end{table}

		Warto jednak zauważyć, iż nawet ten model po dostrojeniu parametrów potrafi wygenerować czytelne tematy.
		Oznacza to, że część merytorycznych informacji została przeanalizowana podczas generowania wektorów.
		Są one wydobywane, jeśli parametry UMAP oraz HDBSCAN zostaną dostrojone tak,
			aby brane pod uwagę były tylko lokalne zależności (niskie \verb|n_neighbors| oraz \verb|min_samples|).
		Wraz ze wzrostem wartości tych parametrów, brane są pod uwagę ogólniejsze podobieństwa,
			do w przypadku takiego modelu oznacza formalne przywitania, czy nazwiska posłów.
		Jednak nawet najlepszy model (indeks \textbf{34}) jako szum oznaczył prawie 165 tysięcy wypowiedzi, podczas gdy mediana wynosi ok.~137 tysięcy.
		Wynika to z tego, że nawet jeśli model był w stanie wyciągnąć informacje z początku wypowiedzi, to nie zawsze było to wystarczające, przez co więcej dokumentów zostało zaklasyfikowanych jako szum.
		Średni wynik tego modelu wynoszący w zaokrągleniu \(0.45\) również jest znacznie poniżej mediany \(0.57\).

		Zdecydowanie lepiej sprawdziły się warianty modelu, które w pierwszej kolejności dzieliły wypowiedź na zdania, a następnie łączyły wektory wygenerowane dla każdego zdania osobno.
		Najlepszy wynik osiągnął model ważący zdania algorytmem LexRank (10 najwyższych wyników przedstawiono w tabeli~\ref{tab:sbert_top}).
		Z trzech analizowanych metod podziału na zdania, metoda oparta o ważenie wektorami tf-idf wypadła najgorzej, co sugeruje iż metoda ta powoduje zbyt dużą utratę informacji.
		Na podstawie wyników nie można również jednoznacznie stwierdzić, iż ważenie wektorów algorytmem LexRank znacząco podnosi jakość wektorów.
		Co prawda najlepszy wynik osiągnął algorytm ważony tym sposobem,
			jednak mediana wyników najwyższa jest dla algorytmu \emph{sbert-mean} (0.592 vs 0.579 dla \emph{sbert-lexrank-weighted} oraz 0.565 dla \emph{sbert-tf-idf-weighted}).
		Wyniki te skłaniają ku wnioskom, że próby ważenia wektorów skutkują utratą informacji, a nie wytłumianiem szumu, jak pierwotnie zakładano.
		Jak pokazują wyniki na wykresie~\ref{fig:consistency_score}, wektory ważone algorytmem tf-idf osiągają znacznie gorsze wyniki,
			podczas gdy uśrednione i ważone algorytmem LexRank są do siebie zbliżone.
		W pozostałych metrykach wszystkie trzy metody osiągały podobne wyniki co oznacza, że ważenie wektorów w najlepszym wypadku jedynie nie pogarsza wyniku.
		
		W przypadku modeli \emph{top1} traktujących wektor o najwyższym wyniku (lub średnią pięciu wektorów --- \emph{top5}) wg.\ rankingu LexRank lub tfidf,
			wyniki były znacznie gorsze od pozostałych ze względu na utratę zbyt dużej ilości danych.
		Podobnych przyczyn (aczkolwiek w mniejszym stopniu) można się doszukiwać w przyczynach gorszej dokładności modeli ważonych.
		Okazuje się, że każda informacja zawarta w wektorach zdań może okazać się istotna i podczas analizy wielkiej ilości danych,
			informacje te mogą okazać się istotne.
		Z tego powodu najlepsze wyniki osiągnął model \emph{sbert-mean}, który uśrednia wektory wszystkich zdań bez wag.

		\begin{table}[htb]
			\caption{Wyniki 10 najlepszych modeli SBERT}\label{tab:sbert_top}
			\centering
			\begin{tabular}{llrrrr}
				\toprule
				{}  &	\emph{model} &  \emph{n\_neighbors} &  \emph{min\_cluster\_size} &  \emph{min\_samples} &   \emph{avg} \\
				\midrule
				118 &  sbert-lexrank-weighted &           15 &               100 &            1 &  0.679364 \\
				202 &              sbert-mean &           15 &               100 &            1 &  0.658695 \\
				92  &  sbert-lexrank-weighted &            5 &               100 &           50 &  0.653523 \\
				201 &              sbert-mean &           15 &                50 &           50 &  0.647214 \\
				285 &   sbert-tf-idf-weighted &           15 &                50 &           50 &  0.642425 \\
				230 &              sbert-mean &           25 &               100 &            1 &  0.640095 \\
				107 &  sbert-lexrank-weighted &           10 &               100 &          100 &  0.638268 \\
				187 &              sbert-mean &           10 &                50 &           50 &  0.636077 \\
				191 &              sbert-mean &           10 &               100 &          100 &  0.635758 \\
				132 &  sbert-lexrank-weighted &           20 &               100 &            1 &  0.632976 \\
				\bottomrule
				\end{tabular}
		\end{table}

	\subsection{TF-IDF}\label{sec:tfidf_summary}
		Metoda tworzenia wektorów za pomocą algorytmu TF-IDF okazała się skuteczna do analizy danych z korpusu.
		Skuteczność tego algorytmu podyktowana jest specyfiką problemu.
		Rozmiar analizowanych wypowiedzi, który jest źródłem problemów dla algorytmów opartych o uczenie maszynowe,
			jednocześnie sprzyja dokładności algorytmów statystycznych.
		Im dłuższe są analizowane dokumenty, tym większa różnorodność występujących słów i większa szansa na znalezienie podobnych słów w wypowiedziach na ten sam temat.
		Jak pokazano na rys.~\ref{fig:topic_counts}, analiza wektorów tf-idf skutkowała wykryciem znacznie większej liczby tematów, niż w przypadku pozostałych metod.
		Obserwacja ta jest zgodna z oczekiwaniami, ponieważ ze względu na metodykę, w której nie ma pojęcia podobieństwa słów,
			wiele dokumentów nie ma wspólnych tokenów pomimo podobnej tematyki (z powodu różnych odmian tych samych słów).
		Jeśli więc już wystąpią jakieś wspólne słowa, to powstałe wektory będą blisko siebie w przestrzeni.
		Należy więc oczekiwać, że wektory wygenerowane w ten sposób będą zawierać wiele grup silnie skoncentrowanych.
		
		Jak się jednak okazuje, odpowiednio skonfigurowany algorytm grupowania HDBSCAN potrafi sobie poradzić również z wektorami w takich przestrzeniach.
		Wraz ze wzrostem wartości parametru \emph{min\_cluster\_size}, powstrzymujemy algorytm przed rozbiciem klastrów na pomniejsze grupy
			(podczas etapu kondensowania hierarchii grup opisanego w rozdziale~\ref{sec:hdbscan}).
		W ten sposób możliwa jest redukcja liczby tematów z prawie 1800 (dla \emph{min\_cluster\_size=5}) do mniej niż 200 (dla \emph{min\_cluster\_size=200}).
		Efekt ten jest widoczny na wykresie~\ref{fig:topic_counts}, gdzie najwyższy wynik uzyskał model dla parametrów \emph{min\_cluster\_size=200} oraz \emph{min\_samples=100}.
		Uzyskał on prawie tak wysoki wynik, jak pozostałe modele dla parametrów \emph{min\_cluster\_size=100} oraz \emph{min\_samples=1}.
		Proces ten pozwolił wektorom tf-idf na osiągnięcie lepszych uśrednionych wyników niż jakiekolwiek inne analizowane metody (rys.~\ref{fig:avg_scores}, zakres dla \emph{min\_cluster\_size=200}).

		Tak drastyczna redukcja tematów wiąże się jednak z innym problemem.
		Ze względu na wysoką specyficzność tematów wygenerowanych przez tfidf, grupy powstałe przez połączenie mniejszych grup nie zawsze są spójne.
		Relację tę dobrze ukazuje metryka \(C_{UMass}\) (Rys.~\ref{fig:umass_score}).
		Metryka ta mierzy współwystępowanie słów z reprezentacji tematu we wszystkich dokumentach grupy.
		Oba modele \emph{tfidf} osiągają znacznie gorszy wynik od pozostałych, co wskazuje na duże zróżnicowanie zgrupowanych dokumentów.
		Oznacza to, że wysoki wynik osiągany przez modele \(tfidf\) nie wynika z większej dokładności w wykrywaniu zależności między wypowiedziami.
		Przeciwnie, jest to oznaka mniejszej dokładności modelu, który w mniejszym stopniu wykrywa różnice między dokumentami.
		W tym kontekście nie jest zaskakującym wysoki wynik modeli zmierzony pozostałymi metrykami.

		Zgodnie z oczekiwaniami, wstępne przetworzenie danych poprzez lematyzację przyniosło wymierne polepszenie wyników.
		Średni wynik dla wszystkich parametrów wzrósł z \(0.58\) do \(0.62\),
			podczas gdy wynik dla najlepszych parametrów (modele o indeksach odpowiednio 376 i 457 w tabeli z wynikami) wzrósł z \(0.71\) do \(0.75\).
		Najlepszy wynik uzyskany metodą inną niż tfidf wyniósł \(0.68\) uzyskany ex aequo przez modele \emph{sbert-lexrank-weighted} (indeks 118) oraz \emph{use-default} (indeks 443).
		Przewaga jest więc istotna i widoczna w wynikach.
		
	\subsection{Universal Sentence Encoder}
		Wariant USE testowany w pracy zbudowany jest na konwolucyjnych sieciach neuronowych.
		Architektura ta nie jest zwykle utożsamiana z zadaniami NLP (ang.\ \emph{Natural Language Processing}), jednak również możliwe jest osiągnięcie dobrych rezultatów.
		Tak też było w tym przypadku.
		Ciekawą różnicą w stosunku do modeli opartych o SBERT, wektory wygenerowane przez USE dawały najlepsze wyniki dla modeli analizujących bezpośrednio całą wypowiedź (\emph{use-default}).
		Oznacza to, że model USE dużo lepiej radzi sobie z długimi zdaniami.
		Różnica ta obrazuje, na jak krótkich zdaniach wytrenowany został model SBERT\@.
		Pomimo większego limitu wejścia tych modeli (512 tokenów), model operujący na całych wypowiedziach (\emph{sbert-default}) osiągał najsłabsze wyniki.
		Tymczasem model USE najlepsze wyniki osiągał na całych wypowiedziach pomimo bardziej ograniczającego limitu 256 tokenów.
		Wysoka dokładność USE może też być efektem znanego zjawiska (zwykle w kontekście artykułów czy wiadomości),
			gdzie na początku dokumentu zawarta jest często informacja, o czym jest dokument (jako wstęp).
		Możliwe, że wypowiedzi posłów w sejmie również często posiadają analogiczną strukturę
			(na wstępie odniesienie się do tego, o czym jest mowa, a następnie rozwinięcie szczegółów), co ułatwia przyporządkowanie tematów na podstawie samego początku dokumentu.

		Jak pokazano na podsumowaniu w tabeli~\ref{tab:avg_scores}b, wektory wygenerowane przez USE dla każdego zdania z osobna, osiągają istotnie gorsze wyniki,
			jeśli zastosowane zostało ważenie wektorów.
		Użycie zwykłej średniej wektorów daje znacznie lepsze wyniki, aczkolwiek różnica między tymi dwoma podejściami jest diametralna
			(w przeciwieństwie do wektorów SBERT, gdzie ważone wektory są niewiele gorsze).
	
\section{Porównanie z LDA}
	Tematy wygenerowane przez LDA okazały się mało użyteczne ze względu na niską jakość reprezentacji (rozdział~\ref{sec:lda}).
	Niemniej warto jednak przyjrzeć się wynikom funkcji \(C_{NPMI}\) i \(C_{UMass}\) ukazanym na rys.~\ref{fig:lda_scores}.
	Obie te miary oceniły najwyżej modele zawierające do 80 tematów.
	Jak wykazała manualna analiza, reprezentacje tych tematów były nieporównywalnie gorsze od reprezentacji tematów wygenerowanych przez BERTopic.
	Pomimo to, model LDA dla 80 tematów osiąga wynik ok.~0.068 (tabela~\ref{tab:lda_scores}),
		podczas gdy tak wysoką wartość osiąga jedynie kilka modeli wygenerowanych przez BERTopic (np.~model tfidf o indeksie 367).
	Jest to znak, że metryka ta może nie być dobrym wyznacznikiem jakości tematów dla tego korpusu.

	Metryka \(C_{UMass}\) wydaje się być bardziej spójna dla tego korpusu.
	Model LDA dla 80 tematów osiągnął wartość \(-0.813\).
	Nadal jest to wysoka wartość, jak na jakość reprezentacji tematów, jednak nie jest to wartość nieprzekraczana przez modele BERTopic.
	Porównując reprezentacje tematów obu typów modeli dla tego samego wyniku \(C_{UMass}\) wciąż widoczna jest duża różnica w jakości,
		dlatego metryka ta również nie jest idealnym odwzorowaniem jakości tematów.
	
\section{Omówienie wyników dla poszczególnych metryk}
	Ze względu na naturę nienadzorowanych problemów, żadna metryka nie odzwierciedla w pełni właściwości wyników.
	Każda metryka jest jedynie przybliżeniem i oceną pewnych aspektów rezultatu, jednak nie może być traktowana jako uniwersalna prawda.
	W kolejnych częściach przedstawione zostaną wyniki poszczególnych metryk oraz jakie wiążą się z nimi wady.

	Tabela~\ref{tab:corr} przedstawia macierz korelacji Pearsona między wykorzystywanymi parametrami oraz metrykami.
	Nie zauważono dużej korelacji między wykorzystywanymi metrykami co oznacza,
		że wyciąganie z nich średniej arytmetycznej nie powinno faworyzować poszczególnych aspektów analizowanych danych.
	Jedyna istotna korelacja (\(>0.5\)) wystąpiła pomiędzy metrykami \(U_{MAss}\) oraz \(c_{NPMI}\).
	Jest to odwrotna korelacja (\(-0.801\)), co sugeruje, że metryki te zupełnie inaczej oceniają spójność wykrytych tematów.

	\begin{table}[htb]
		\caption{Macierz korelacji Pearsona dla parametrów oraz metryk}\label{tab:corr}
		\centering
		\small
		\begin{tabular}{c|rrrrrrrrr}	% chktex 44
			\toprule
			{} &  \rotatebox[origin=c]{90}{\emph{n\_neighbors}} & \rotatebox[origin=c]{90}{\emph{min\_cluster\_size}} & \rotatebox[origin=c]{90}{\emph{min\_samples}} & \rotatebox[origin=c]{90}{\(C_{UMass}\)} & \rotatebox[origin=c]{90}{\(C_{NPMI}\)} & \rotatebox[origin=c]{90}{\(s_c\)} & \rotatebox[origin=c]{90}{\(s_f\)} & \rotatebox[origin=c]{90}{\(s_T\)} & \rotatebox[origin=c]{90}{\(avg\)} \\
			\midrule
			\emph{n\_neighbors}					&  1.000 &  0.006 &  0.004 &  0.111 &  0.018 &  0.054 & -0.187 &  0.102 &  0.071 \\
			\emph{min\_cluster\_size}		&  0.006 &  1.000 &  0.469 &  0.662 & -0.557 &  0.108 &  0.506 &  0.381 &  0.514 \\
			\emph{min\_samples}					&  0.004 &  0.469 &  1.000 &  0.446 & -0.260 &  0.043 &  0.293 &  0.120 &  0.289 \\
			\(C_{UMass}\)								&  0.111 &  0.662 &  0.446 &  1.000 & -0.801 & -0.216 &  0.065 &  0.491 &  0.280 \\
			\(C_{NPMI}\)								&  0.018 & -0.557 & -0.260 & -0.801 &  1.000 &  0.346 &  0.166 & -0.380 &  0.098 \\
			\(s_c\)											&  0.054 &  0.108 &  0.043 & -0.216 &  0.346 &  1.000 &  0.426 & -0.088 &  0.626 \\
			\(s_f\)											& -0.187 &  0.506 &  0.293 &  0.065 &  0.166 &  0.426 &  1.000 & -0.001 &  0.686 \\
			\(s_T\)											&  0.102 &  0.381 &  0.120 &  0.491 & -0.380 & -0.088 & -0.001 &  1.000 &  0.536 \\
			\(avg\)											&  0.071 &  0.514 &  0.289 &  0.280 &  0.098 &  0.626 &  0.686 &  0.536 &  1.000 \\
			\bottomrule
		\end{tabular}			
	\end{table}

	\subsection{Spójność reprezentacji tematów}
		Wyniki testów metryką \(C_{UMass}\) przedstawione zostały na rysunku~\ref{fig:umass_score} oraz w tabeli z wynikami w rozdziale~\ref{sec:all_scores}.
		Wykres przedstawia dane dla ograniczonego zakresu parametrów (stałe \emph{n\_neighbors} równe 15), jednak podobne zależności występują dla pozostałych wartości parametru.
		Zaskakującym może się wydawać dobry wynik \emph{sbert-default}, jednak jak opisano w rozdziale~\ref{sec:sbert_summary},
			model ten generuje bardzo ogólne wektory, ponieważ analizuje jedynie początki wypowiedzi.
		Metryka \(C_{UMass}\) bada współwystępowanie słów tematu w klastrach, co w naturalny sposób promuje ogólne tematy,
			takie jak te generowane przez model \emph{sbert-default}.
		Ważnym fragmentem wykresu są wyniki modeli \emph{tfidf} i \emph{tfidf-spacy}.
		Są one znacznie gorsze od pozostałych, co wytłumaczone zostało w podrozdziale~\ref{sec:tfidf_summary}.
		Pozostałe metody osiągnęły podobne wyniki.
		
		Na wykresie widoczna jest również korelacja metryki z parametrem \emph{min\_cluster\_size}, która wynosi \(0.662\) (tabela~\ref{tab:corr}).
		Wraz ze wzrostem minimalnego rozmiaru grupy, maleje liczba tematów, a ich rozmiar rośnie.
		Zwiększa się też zatem ich zróżnicowanie, przez co tematy stają się bardziej ogólne.
		Coraz lepszy wynik wraz ze wzrostem minimalnego rozmiaru grupy jest więc prawdopodobnie spowodowany tym samym czynnikiem,
			co niespodziewany wysoki wynik modelu \emph{sbert-default}.

		Metryka \(C_{NPMI}\), podobnie jak \(C_{UMass}\) bada spójność reprezentacji tematu w kontekście zawartości dokumentów z klastra.
		Te wyniki prezentują się zgoła odmiennie.
		Model \emph{sbert-default} zgodnie z oczekiwaniami osiąga tym razem najgorsze wyniki.
		Najwyższe wyniki osiągnięte zostały przez modele bazujące na tf-idf.
		Kolejnym najlepszym algorytmem odstającym od reszty okazał się \emph{use-tf-idf-weighted}.
		Jest to o tyle ciekawe, że te wektory osiągały gorsze wyniki od pozostałych we wszystkich innych testach.

		Różnica w wynikach pomiędzy \(C_{UMass}\), a \(C_{NPMI}\) oraz ujemna korelacja między nimi podkreśla fakt, że algorytmy te są jedynie pośrednim przybliżeniem spójności tematów.
		Jedyną metodą dającą pewność co do wyników, jest manualna weryfikacja przez człowieka.
		Ze względu na ograniczone zasoby metoda ta nie mogła jednak zostać zastosowana w tej pracy.

	\subsection{Spójność rozkładu tematów}
		Algorytm oznaczony jako \(s_c\) bada, czy przyporządkowanie tematów jest spójne ze strukturą przebiegu posiedzeń plenarnych sejmu.
		Na wykresie~\ref{fig:consistency_score} można zauważyć, iż metryka ta jest mniej zależna od wybranych parametrów.
		Dla niskich wartości \emph{n\_neighbors}, które skutkują dużą liczbą tematów, otrzymano znacznie gorsze wyniki.
		Jednak już od \emph{n\_neighbors} równego 10, wyniki dla większości modeli ustabilizowały się.
		Jedynie modele oparte o tfidf lekko poprawiały swoje wyniki do samego końca, co jest spójne z obserwacjami w pozostałych testach.

		Najwyższe wyniki osiągnął model \emph{tfidf-spacy}, który operuje na danych poddanych lematyzacji.
		Warte uwagi jest to, że najwyższy wynik nie został osiągnięty przez model, który przyporządkował najwięcej dokumentów.
		Oznacza to, że metryka jest w pewnym stopniu odporna na modele tworzące zbyt ogólne tematy poprzez niską dokładność.
		Niemniej macierz korelacji wskazuje na istniejącą korelację pomiędzy wynikami \(s_c\), a \(s_f\), która wyniosła \(0.426\).
		Nie jest to jednak wystarczająco wysoka wartość by stwierdzić, że wartości te są od siebie zależne.

		Patrząc na tabelę z wynikami, modele, które wykryły od kilku do kilkunastu tematów, zgodnie z oczekiwaniami osiągnęły bardzo wysoki wynik w tej mierze.
		Należy mieć to na uwadze, że miara ta nie może posłużyć do oceny takich modeli.
		To samo tyczy się również poprzednich dwóch miar, gdzie dla \(C_{UMass}\) modele te zbliżyły się do wartości równej 0 (najlepszy możliwy wynik),
			a dla \(C_{NPMI}\) osiągały wyniki dwu-trzykrotnie lepsze niż najlepsze poprawne modele.

	\subsection{Pozostałe}
		Metryka związana z liczbą przyporządkowanych dokumentów przyniosła spodziewane rezultaty.
		Im bardziej ogólny jest model, tym lepszy uzyskał wynik.
		Jest to bardzo prosta metoda i sama w sobie nie może stanowić żadnej oceny wartości modeli.
		Została jednak uwzględniona jako składowa oceny uśredniającej wyniki, ponieważ jest to istotna informacja o uzyskanych klastrach.
		Jeśli model przyporządkuje małą liczbę dokumentów, to miara taka jak \(C_{UMass}\) uzna uzyskane grupowania za spójne i przypisze odpowiednio wysoki wynik.
		Miara ta pełni więc rolę filtra, który ma za zadanie obniżać wyniki tych modeli, które pominęły zbyt dużą liczbę dokumentów (przykładowo \emph{sbert-default}).
		Najlepsze wyniki ponownie osiągnęły modele bazujące na tfidf, co również było spodziewane ze względu na ich ograniczoną dokładność.

		Analiza liczby odnalezionych tematów pełni podobną rolę.
		Modele, które odnajdują zbyt dużą liczbę tematów (\emph{tfidf, tfidf-spacy}) zawdzięczają wysokie wyniki dużej liczbie grup.
		Dzięki znormalizowaniu liczby odnalezionych tematów, możliwa jest ocena danego modelu pod tym kątem.
		Zarówno modele odnajdujące za mało tematów (np.~modele \emph{sbert} oraz \emph{use} dla \emph{min\_cluster\_size=200}) otrzymują niski wynik.
		Analogicznie modele, które wykryły za dużo tematów, otrzymują równie niski wynik.
		Zależność ta przedstawiona została na rysunku~\ref{fig:topic_counts}.
		
		Warto zauważyć, iż modele oparte na \emph{tfidf} osiągnęły wysokie wyniki w obu omawianych wynikach dla parametru \emph{min\_cluster\_size=200}.
		Pozostałe metody dla tak wysokiej minimalnej liczby tematów osiągały gorsze wyniki niż dla mniejszych wartości.
		W połączeniu z najwyższymi wynikami we wszystkich pozostałych metrykach oprócz \(C_{UMass}\), pozwoliło to na zajęcie najwyższych miejsc w rankingu.