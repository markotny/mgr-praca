%%%%%%%%%%%%%%%%%%%%%%%%%%%%%%%%%%%%%%%%%%%%%%%%%%%%%%%%%%%%%%%%%%%%%%%%%
%  cSpell:disable
%  Zawartość: Główny plik szablonu pracy dyplomowej (magisterskiej/inżynierskiej). 
%  Opracował: Tomasz Kubik <tomasz.kubik@pwr.edu.pl>
%  Data: 9 lutego 2021
%  Wersja: 0.6
%  Wymagania: dedykowany do kompilatora pdflatex.
%%%%%%%%%%%%%%%%%%%%%%%%%%%%%%%%%%%%%%%%%%%%%%%%%%%%%%%%%%%%%%%%%%%%%%%%%

\documentclass[a4paper,onecolumn,oneside,12pt,extrafontsizes]{memoir}
%  W celu przygotowania wydruku  do archiwum można:
%  a) przygotować pdf, w którym dwie strony zostaną wstawione na jedną fizyczną stronę i taki dokument wydrukować dwustronnie (podejście zalecane)
%
%   Taki dokument można przygotować poprzez
%   - wydruk z Adobe Acrobat Reader z opcją "Wiele" - sekcja "Rozmiar i obsługa stron"
%   - wykorzystanie narzędzi psutils
%
%      Windows (zakładając, że w dystrybucji MiKTeX jest pakiet miktex-psutils-bin-x64-2.9):
%        "c:\Program Files\MiKTeX 2.9\miktex\bin\x64\pdf2ps.exe" Dyplom.pdf Dyplom.ps
%        "c:\Program Files\MiKTeX 2.9\miktex\bin\x64\psnup.exe" -2 Dyplom.ps Dyplom2.ps
%        "c:\Program Files\MiKTeX 2.9\miktex\bin\x64\ps2pdf.exe" Dyplom2.ps Dyplom2.pdf
%        Del Dyplom2.ps Dyplom.ps
%
%     Linux:
%        pdf2ps Dyplom.pdf - | psnup -2 | ps2pdf - Dyplom2.pdf
%
%  b) przekomplilować dokument zmniejszając czcionkę (podejście niezalecane, bo zmienia formatowanie dokumentu)
%
%    Do tego wystarczy posłużyć się poniższymi komendami (zamiast documentclass z pierwszej linijki):
%   \documentclass[a4paper,onecolumn,twoside,10pt]{memoir} 
%   \renewcommand{\normalsize}{\fontsize{8pt}{10pt}\selectfont}

% \usepackage[cp1250]{inputenc} % Proszę zostawić, jeśli kodowanie edytowanych plików to cp1250 
\usepackage[utf8]{inputenc} % Proszę użyć zamiast powyższego, jeśli kodowanie edytowanych plików to UTF8
\usepackage[T1]{fontenc}
\usepackage[english,polish]{babel} % Tutaj ważna jest kolejność atrybutów (dla pracy po polsku polish powinno być na końcu)
%\DisemulatePackage{setspace}
\usepackage{setspace}
\usepackage{color,calc}
%\usepackage{soul} % pakiet z komendami do podkreślania, przekreślania, podświetlania tekstu (raczej niepotrzebny)

\usepackage{ebgaramond} % pakiet z czcionkami garamond, potrzebny tylko do strony tytułowej, musi wystąpić przed pakietem tgtermes

%% Aby uzyskać polskie literki w pdfie (a nie zlepki) korzystamy z pakietu czcionek tgterms. 
%% W pakiecie tym są zdefiniowane klony czcionek Times o kształtach: normalny, pogrubiony, italic, italic pogrubiony.
%% W pakiecie tym brakuje czcionki o kształcie: slanted (podobny do italic). 
%% Jeśli w dokumencie gdzieś zostanie zastosowana czcionka slanted (np. po użyciu komendy \textsl{}), to
%% latex dokona podstawienia na czcionkę standardową i zgłosi to w ostrzeżeniu (warningu).
%% Ponadto tgtermes to czcionka do tekstu. Wszelkie matematyczne wzory będą sformatowane domyślną czcionką do wzorów.
%% Jeśli wzory mają być sformatowane z wykorzystaniem innych czcionek, trzeba to jawnie zadeklarować.

%% Po zainstalowaniu pakietu tgtermes może będzie trzeba zauktualizować informacje 
%% o dostępnych fontach oraz mapy. Można to zrobić z konsoli (jako administrator)
%% initexmf --admin --update-fndb
%% initexmf --admin --mkmaps

\usepackage{tgtermes}   
\renewcommand*\ttdefault{txtt}



\usepackage{multicol} % pakiet umożliwiający stworzenie wielokolumnowego tekstu
%%%%%%%%%%%%%%%%%%%%%%%%%%%%%%%%%%%%%%%%%%%%%%%%%%%
%% Pakiety do formatowania tabel
%%%%%%%%%%%%%%%%%%%%%%%%%%%%%%%%%%%%%%%%%%%%%%%%%%%
\usepackage{tabularx}
% Proszę używać tylko tabularx. Innych pakietów proszę nie stosować !!!
% Dokument na pewno da się zredagować bez ich użycia.
%\usepackage{longtable}
%\usepackage{ltxtable}
%\usepackage{tabulary}

%%%%%%%%%%%%%%%%%%%%%%%%%%%%%%%%%%%%%%%%%%%%%%%%%%%
%% Pakiet do wstawiania fragmentów kodu
%%%%%%%%%%%%%%%%%%%%%%%%%%%%%%%%%%%%%%%%%%%%%%%%%%%
\usepackage{listings} 
\usepackage{xpatch}
\makeatletter
\xpatchcmd\l@lstlisting{1.5em}{0em}{}{}
\makeatother
% Pakiet dostarcza otoczenia lstlisting. Jest ono wysoce konfigurowalne. 
% Konfigurować można indywidualnie każdy z listingów lub globalnie, w poleceniu \lstset{}.

% Zalecane jest, by kod źródłowy był wyprowadzany z użyciem czcionki maszynowej \ttfamily
% Ponieważ kod źródłowy, nawet po obcięciu do interesujących fragmentów, bywa obszerny, należy zmniejszyć czcionkę.
% Zalecane jest \small (dla krótkich fragmentów) oraz \footnotesize (dla dłuższych fragmentów).

% Ponadto podczas konfiguracji można zadeklarować sposób numerowania linii. Numerowanie linii zalecane jest jednak 
% tylko w przypadkach, gdy w opisie zamieszczonym w tekście znajdują się jakieś odwołania do konkretnych linii.
% Jeśli w opisie nie ma takich odwołań, numerowanie linii jest zbędne. Proszę wtedy go nie stosować.
% Przy włączaniu numerowania linii należy zwrócić uwagę na to, gdzie pojawią się te numery.
% Bez zmiany dodatkowych parametrów pojawiają się one na marginesie strony (co jest nieporządane).

\lstset{
  % basicstyle=\small\ttfamily, % lub basicstyle=\footnotesize\ttfamily
	basicstyle =\footnotesize,
  %%columns=fullflexible,
	%%showstringspaces=false,
	%%showspaces=false,
  breaklines=true,
  postbreak=\mbox{\textcolor{red}{$\hookrightarrow$}\space}, 
  numbers=left,  % ta i poniższe linie dotyczą ustawienia numerowania i sposobu jego wyprowadzania
	tabsize=2,
	frame=l,
  %%firstnumber=1, 
  %%numberfirstline=true, 
	%%xleftmargin=17pt,
  %%framexleftmargin=17pt,
  %%framexrightmargin=5pt,
  %%framexbottommargin=4pt,
	belowskip=.5\baselineskip
}

% Jeśli edytowany plik nie jest w kodowaniu cp1250, to jest problem z polskimi znakami występującymi we wstawianym kodzie.
% Dlatego podczas pracy na plikach w kodowaniu UTF-8 trzeba zadeklarować mapowanie jak niżej (wystarczy odmarkować).
% Niestety, jak się zastosuje to mapowanie mogą pojawić się problemy z podświetlaniem składni (patrz dalej).
\lstset{literate=%-
{ą}{{\k{a}}}1 {ć}{{\'c}}1 {ę}{{\k{e}}}1 {ł}{{\l{}}}1 {ń}{{\'n}}1 {ó}{{\'o}}1 {ś}{{\'s}}1 {ż}{{\.z}}1 {ź}{{\'z}}1 {Ą}{{\k{A}}}1 {Ć}{{\'C}}1 {Ę}{{\k{E}}}1 {Ł}{{\L{}}}1 {Ń}{{\'N}}1 {Ó}{{\'O}}1 {Ś}{{\'S}}1 {Ż}{{\.Z}}1 {Ź}{{\'Z}}1 
    {Ö}{{\"O}}1
    {Ä}{{\"A}}1
    {Ü}{{\"U}}1
    {ß}{{\ss}}1
    {ü}{{\"u}}1
    {ä}{{\"a}}1
    {ö}{{\"o}}1
    {~}{{\textasciitilde}}1
		{—}{{{\textemdash} }}1
}%{\ \ }{{\ }}1}


%% lstlisting pozwala na ostylowania podświetlania składni wybranych języków.
%% Działa to na zasadzie zdefiniowania słów kluczowych oraz sposobu ich wyświetlania.
%% Ponieważ jest to prosty mechanizm, czasem trudno osiągnąć takie efekty, jakie dają narzędzia IDE. 
%% Jednak w większości przypadku osiągane rezutlaty są zadowalające.
%%\lstloadlanguages{% Check Dokumentation for further languages ...
%%C,
%%C++,
%%csh,
%%Java
%%}


%% lstlisting obsługuje domyślnie kilka najpopularniejszych języków.
%% Inne języki muszą być dodefiniowane. Poniżej podano przykłady definicji języków i styli.

\definecolor{lightgray}{rgb}{.9,.9,.9}
\definecolor{darkgray}{rgb}{.4,.4,.4}
\definecolor{purple}{rgb}{0.65, 0.12, 0.82}
\definecolor{javared}{rgb}{0.6,0,0} % for strings
\definecolor{javagreen}{rgb}{0.25,0.5,0.35} % comments
\definecolor{javapurple}{rgb}{0.5,0,0.35} % keywords
\definecolor{javadocblue}{rgb}{0.25,0.35,0.75} % javadoc
 
\lstdefinelanguage{JavaScript}{ 
	keywords={typeof, new, true, false, catch, function, return, null, catch, switch, var, if, in, while, do, else, case, break},
	keywordstyle=\color{blue}\bfseries,
	ndkeywords={class, export, boolean, throw, implements, import, this},
	ndkeywordstyle=\color{darkgray}\bfseries,
	identifierstyle=\color{black},
	sensitive=false,
	comment=[l]{//},
	morecomment=[s]{/*}{*/},
	commentstyle=\color{purple}\ttfamily,
	stringstyle=\color{red}\ttfamily,
	morestring=[b]',
	morestring=[b]"
}
\lstdefinestyle{JavaScriptStyle}{
	language=JavaScript,
	commentstyle=\color{javagreen}, % niestety, jeśli w linii komentarza pojawią się słowa kluczowe, to zostaną pokolorowane
	backgroundcolor=,%\color{lightgray}, % można ustwić kolor tła, ale jest to niezalecane
	extendedchars=true,
	basicstyle=\footnotesize\ttfamily,
	showstringspaces=false,
	showspaces=false,
	numbers=none,%left,
	numberstyle=\footnotesize,
	numbersep=9pt,
	tabsize=2,
	breaklines=true,
	showtabs=false,
	captionpos=t
}

\lstdefinestyle{JavaStyle}{
basicstyle=\footnotesize\ttfamily,
keywordstyle=\color{javapurple}\bfseries,
stringstyle=\color{javared},
commentstyle=\color{javagreen},
morecomment=[s][\color{javadocblue}]{/**}{*/},
numbers=none,%left,
numberstyle=\tiny\color{black},
stepnumber=2,
numbersep=10pt,
tabsize=4,
showspaces=false,
showstringspaces=false,
captionpos=t
}

\definecolor{pblue}{rgb}{0.13,0.13,1}
\definecolor{pgreen}{rgb}{0,0.5,0}
\definecolor{pred}{rgb}{0.9,0,0}
\definecolor{pgrey}{rgb}{0.46,0.45,0.48}
\definecolor{dark-grey}{rgb}{0.4,0.4,0.4}
% styl json
\newcommand\JSONnumbervaluestyle{\color{blue}}
\newcommand\JSONstringvaluestyle{\color{red}}

\newif\ifcolonfoundonthisline

\makeatletter

\lstdefinestyle{json-style}  
{
	showstringspaces    = false,
	keywords            = {false,true},
	alsoletter          = 0123456789.,
	morestring          = [s]{"}{"},
	stringstyle         = \ifcolonfoundonthisline\JSONstringvaluestyle\fi,
	MoreSelectCharTable =%
	\lst@DefSaveDef{`:}\colon@json{\processColon@json},
	basicstyle          = \footnotesize\ttfamily,
	keywordstyle        = \ttfamily\bfseries,
	numbers				= left, % zamarkować, jeśli numeracja linii jest niepotrzebna
	numberstyle={\footnotesize\ttfamily\color{dark-grey}},
	xleftmargin			= 2em % zamarkować, jeśli numeracja linii jest niepotrzebna
}

\lstdefinestyle{algorithm}{
	mathescape=true,
	frame=tB,
	numbers=left, 
	numberstyle=\tiny,
	basicstyle=\scriptsize, 
	keywordstyle=\color{black}\bfseries\em,
	keywords={,input, output, return, datatype, function, in, if, else, for, each, while, begin, end, do, }
	numbers=left,
	xleftmargin=.04\textwidth,
	postbreak=\space
}


\newcommand\processColon@json{%
	\colon@json%
	\ifnum\lst@mode=\lst@Pmode%
	\global\colonfoundonthislinetrue%
	\fi
}

\lst@AddToHook{Output}{%
	\ifcolonfoundonthisline%
	\ifnum\lst@mode=\lst@Pmode%
	\def\lst@thestyle{\JSONnumbervaluestyle}%
	\fi
	\fi
	\lsthk@DetectKeywords% 
}

\lst@AddToHook{EOL}%
{\global\colonfoundonthislinefalse}

\makeatother

%%\definecolor{red}{rgb}{0.6,0,0} % for strings
%%\definecolor{blue}{rgb}{0,0,0.6}
%%\definecolor{green}{rgb}{0,0.8,0}
%%\definecolor{cyan}{rgb}{0.0,0.6,0.6}
%%
%%\lstdefinestyle{sqlstyle}{
%%language=SQL,
%%basicstyle=\footnotesize\ttfamily, 
%%numbers=left, 
%%numberstyle=\tiny, 
%%numbersep=5pt, 
%%tabsize=2, 
%%extendedchars=true, 
%%breaklines=true, 
%%showspaces=false, 
%%showtabs=true, 
%%xleftmargin=17pt,
%%framexleftmargin=17pt,
%%framexrightmargin=5pt,
%%framexbottommargin=4pt,
%%keywordstyle=\color{blue}, 
%%commentstyle=\color{green}, 
%%stringstyle=\color{red}, 
%%}
%%
%%\lstdefinestyle{sharpcstyle}{
%%language=[Sharp]C,
%%basicstyle=\footnotesize\ttfamily, 
%%numbers=left, 
%%numberstyle=\tiny, 
%%numbersep=5pt, 
%%tabsize=2, 
%%extendedchars=true, 
%%breaklines=true, 
%%showspaces=false, 
%%showtabs=true, 
%%xleftmargin=17pt,
%%framexleftmargin=17pt,
%%framexrightmargin=5pt,
%%framexbottommargin=4pt,
%%morecomment=[l]{//}, %use comment-line-style!
%%morecomment=[s]{/*}{*/}, %for multiline comments
%%showstringspaces=false, 
%%morekeywords={  abstract, event, new, struct,
                %%as, explicit, null, switch,
                %%base, extern, object, this,
                %%bool, false, operator, throw,
                %%break, finally, out, true,
                %%byte, fixed, override, try,
                %%case, float, params, typeof,
                %%catch, for, private, uint,
                %%char, foreach, protected, ulong,
                %%checked, goto, public, unchecked,
                %%class, if, readonly, unsafe,
                %%const, implicit, ref, ushort,
                %%continue, in, return, using,
                %%decimal, int, sbyte, virtual,
                %%default, interface, sealed, volatile,
                %%delegate, internal, short, void,
                %%do, is, sizeof, while,
                %%double, lock, stackalloc,
                %%else, long, static,
                %%enum, namespace, string},
%%keywordstyle=\color{cyan},
%%identifierstyle=\color{red},
%%stringstyle=\color{blue}, 
%%commentstyle=\color{green},
%%}


% Przedefiniowanie komendy wstawiającej Spis listingów do dokumentu. 
%\renewcommand\lstlistlistingname{Spis listingów}
% Bez przedefiniowania nagłówek tego spisu pojawiał się na innej wysokości niż nagłówki innych spisów.
%%%\makeatletter
%%%%\renewcommand*{\l@lstlisting}[2]{\@dottedtocline{1}{0em}{2.3em}{#1}{#2}}
%%%\g@addto@macro\insertchapterspace{\addtocontents{lol}{\protect\addvspace{10pt}}}
%%%\renewcommand*{\l@lstlisting}{\@dottedtocline{1}{0em}{2.3em}}
%%%\makeatother


\makeatletter % odstępy w spisie pomiędzy rozdziałami
\renewcommand*{\insertchapterspace}{%
  \addtocontents{lof}{\protect\addvspace{5pt}}%
  \addtocontents{lot}{\protect\addvspace{5pt}}%
	\addtocontents{toc}{\protect\addvspace{5pt}} %
  \addtocontents{lol}{\protect\addvspace{5pt}}}
\makeatother 

\newcommand{\listingcaption}[1]% dodane, by można było robić podpis nad dwukolumnowym listingiem
{%
\vspace*{\abovecaptionskip}\small 
\refstepcounter{lstlisting}\hfill%
Listing \thelstlisting: #1\hfill%\hfill%
\addcontentsline{lol}{lstlisting}{\protect\numberline{\thelstlisting}#1}
}%

% Redefinitions of labels for tables, figures and bibliography 
%\AtBeginDocument{% 
        \addto\captionspolish{% 
        \renewcommand{\lstlistlistingname}{Spis listingów}%
}%} 
\newlistof{lstlistoflistings}{lol}{\lstlistlistingname}


%%%%%%%%%%%%%%%%%%%%%%%%%%%%%%%%%%%%%%%%%%%%%%%%%%%
%% Ustawienia odpowiedzialne za sposób łamania dokumentu
%% i ułożenie elementów pływających
%%%%%%%%%%%%%%%%%%%%%%%%%%%%%%%%%%%%%%%%%%%%%%%%%%%
%\hyphenpenalty=10000		% nie dziel wyrazów zbyt często
\clubpenalty=10000      %kara za sierotki
\widowpenalty=10000  % nie pozostawiaj wdów
%\brokenpenalty=10000		% nie dziel wyrazów między stronami - trzeba było wyłączyć, bo nie łamały się linie w lstlisting
%\exhyphenpenalty=999999		% nie dziel słów z myślnikiem - trzeba było wyłączyć, bo nie łamały się linie w lstlisting
\righthyphenmin=3			% dziel minimum 3 litery

%\tolerance=4500
%\pretolerance=250
%\hfuzz=1.5pt
%\hbadness=1450

\renewcommand{\topfraction}{0.95}
\renewcommand{\bottomfraction}{0.95}
\renewcommand{\textfraction}{0.05}
\renewcommand{\floatpagefraction}{0.35}

%%%%%%%%%%%%%%%%%%%%%%%%%%%%%%%%%%%%%%%%%%%%%%%%%%%
%%  Ustawienia rozmiarów: tekstu, nagłówka i stopki, marginesów
%%  dla dokumentów klasy memoir 
%%%%%%%%%%%%%%%%%%%%%%%%%%%%%%%%%%%%%%%%%%%%%%%%%%%
\setlength{\headsep}{10pt} 
\setlength{\headheight}{13.6pt} % wartość baselineskip dla czcionki 11pt tj. \small wynosi 13.6pt
\setlength{\footskip}{\headsep+\headheight}
\setlength{\uppermargin}{\headheight+\headsep+1cm}
\setlength{\textheight}{\paperheight-\uppermargin-\footskip-1.5cm}
\setlength{\textwidth}{\paperwidth-5cm}
\setlength{\spinemargin}{2.5cm}
\setlength{\foremargin}{2.5cm}
\setlength{\marginparsep}{2mm}
\setlength{\marginparwidth}{2.3mm}
%\settrimmedsize{297mm}{210mm}{*}
%\settrims{0mm}{0mm}	
\checkandfixthelayout[fixed] % konieczne, aby się dobrze wszystko poustawiało
%%%%%%%%%%%%%%%%%%%%%%%%%%%%%%%%%%%%%%%%%%%%%%%%
%%  Ustawienia odległości linii, wcięć, odstępów
%%%%%%%%%%%%%%%%%%%%%%%%%%%%%%%%%%%%%%%%%%%%%%%%
\linespread{1}
%\linespread{1.241}
\setlength{\parindent}{14.5pt}

\setlength{\cftbeforechapterskip}{0pt} % odstępy w spisie treści przed rozdziałem, działa w korelacji z:
\renewcommand{\aftertoctitle}{\afterchaptertitle\vspace{-4pt}} % co trzeba zmienić, by tekst w spisie treści zaczynał się na tej samej wysokości co w innych spisach

%\cftsetindents{section}{1.5em}{2.3em}

%\setbeforesecskip{10pt plus 0.5ex}%{-3.5ex \@plus -1ex \@minus -.2ex}
%\setaftersecskip{10pt plus 0.5ex}%\onelineskip}
%\setbeforesubsecskip{8pt plus 0.5ex}%{-3.5ex \@plus -1ex \@minus -.2ex}
%\setaftersubsecskip{8pt plus 0.5ex}%\onelineskip}
%\setlength\floatsep{6pt plus 2pt minus 2pt} 
%\setlength\intextsep{12pt plus 2pt minus 2pt} 
%\setlength\textfloatsep{12pt plus 2pt minus 2pt} 

%%%%%%%%%%%%%%%%%%%%%%%%%%%%%%%%%%%%%%%%%%%%%%%%%%%
%%  Pakiety i komendy zastosowane tylko do zamieszczenia informacji o użytych komendach i fontach w tym szablonie.
%%  Normalnie nie są one potrzebne. Proszę poniższe deklaracje zamarkować podczas redakcji pracy !!!!
%%%%%%%%%%%%%%%%%%%%%%%%%%%%%%%%%%%%%%%%%%%%%%%%%%%
\usepackage{memlays}     % extra layout diagrams, zastosowane w szblonie do 'debuggowania', używa pakietu layouts
%\usepackage{layouts}
\usepackage{printlen} % pakiet do wyświetlania wartości zdefiniowanych długości, stosowany do 'debuggowania'
\uselengthunit{pt}
\makeatletter
\newcommand{\showFontSize}{\f@size pt} % makro wypisujące wielkość bieżącej czcionki
\makeatother
% do pokazania ramek można byłoby użyć:
%\usepackage{showframe} 


%%%%%%%%%%%%%%%%%%%%%%%%%%%%%%%%%%%%%%%%%%%%%%%%%%%
%%  Formatowanie list wyliczeniowych, wypunktowań i własnych otoczeń
%%%%%%%%%%%%%%%%%%%%%%%%%%%%%%%%%%%%%%%%%%%%%%%%%%%

% Domyślnie wypunktowania mają zadeklatorowane znaki, które nie występują w tgtermes
% Aby latex nie podstawiał w ich miejsca znaków z czcionki standardowej można zrobić podstawienie:
%    \DeclareTextCommandDefault{\textbullet}{\ensuremath{\bullet}}
%    \DeclareTextCommandDefault{\textasteriskcentered}{\ensuremath{\ast}}
%    \DeclareTextCommandDefault{\textperiodcentered}{\ensuremath{\cdot}}
% Jednak jeszcze lepszym pomysłem jest zdefiniowanie otoczeń z wykorzystaniem enumitem
\usepackage{enumitem} % pakiet pozwalający zarządzać formatowaniem list wyliczeniowych
\setlist{noitemsep,topsep=4pt,parsep=0pt,partopsep=4pt,leftmargin=*} % zadeklarowane parametry pozwalają uzyskać 'zwartą' postać wypunktowania bądź wyliczenia
\setenumerate{labelindent=0pt,itemindent=0pt,leftmargin=!,label=\arabic*.} % można zmienić \arabic na \alph, jeśli wyliczenia mają być z literkami
\setlistdepth{4} % definiujemy głębokość zagnieżdżenia list wyliczeniowych do 4 poziomów
\setlist[itemize,1]{label=$\bullet$}  % definiujemy, jaki symbol ma być użyty w wyliczeniu na danym poziomie
\setlist[itemize,2]{label=\normalfont\bfseries\textendash}
\setlist[itemize,3]{label=$\ast$}
\setlist[itemize,4]{label=$\cdot$}
\renewlist{itemize}{itemize}{4}

%%%http://tex.stackexchange.com/questions/29322/how-to-make-enumerate-items-align-at-left-margin
%\renewenvironment{enumerate}
%{
%\begin{list}{\arabic{enumi}.}
%{
%\usecounter{enumi}
%%\setlength{\itemindent}{0pt}
%%\setlength{\leftmargin}{1.8em}%{2zw} % 
%%\setlength{\rightmargin}{0zw} %
%%\setlength{\labelsep}{1zw} %
%%\setlength{\labelwidth}{3zw} % 
%\setlength{\topsep}{6pt}%
%\setlength{\partopsep}{0pt}%
%\setlength{\parskip}{0pt}%
%\setlength{\parsep}{0em} % 
%\setlength{\itemsep}{0em} % 
%%\setlength{\listparindent}{1zw} % 
%}
%}{
%\end{list}
%}

\makeatletter
\renewenvironment{quote}{
	\begin{list}{}
	{
	\setlength{\leftmargin}{1em}
	\setlength{\topsep}{0pt}%
	\setlength{\partopsep}{0pt}%
	\setlength{\parskip}{0pt}%
	\setlength{\parsep}{0pt}%
	\setlength{\itemsep}{0pt}
	}
	}{
	\end{list}}
\makeatother

%%%%%%%%%%%%%%%%%%%%%%%%%%%%%%%%%%%%%%%%%
%%  Pakiet i komendy do generowania indeksu 
%% (ważne, by pojawiły się przed pakietem hyperref)
%%%%%%%%%%%%%%%%%%%%%%%%%%%%%%%%%%%%%%%%%
% pdftex jest w stanie wygenerować indeks (czyli spis haseł z referencjami do stron, na których te hasła się pojawiły).
% Generalnie z indeksem jest sporo problemów, zwłaszcza, gdy pojawiają się polskie literki.
% Trzeba wtedy korzystać z xindy.
% Zwykle w pracach dyplomowych indeksy nie są wykorzystywane. Dlatego są zamarkowane.
%\DisemulatePackage{imakeidx}
%\usepackage[makeindex,noautomatic]{imakeidx} % tutaj mówimy, żeby indeks nie generował się automatycznie, 
%\makeindex
%
%\makeatletter
%%%%\renewenvironment{theindex}
							 %%%%{\vskip 10pt\@makeschapterhead{\indexname}\vskip -3pt%
								%%%%\@mkboth{\MakeUppercase\indexname}%
												%%%%{\MakeUppercase\indexname}%
								%%%%\vspace{-3.2mm}\parindent\z@%
								%%%%\renewcommand\subitem{\par\hangindent 16\p@ \hspace*{0\p@}}%%
								%%%%\phantomsection%
								%%%%\begin{multicols}{2}
								%%%%%\thispagestyle{plain}
								%%%%\parindent\z@                
								%%%%%\parskip\z@ \@plus .3\p@\relax
								%%%%\let\item\@idxitem}
							 %%%%{\end{multicols}\clearpage}
%%%%
%\makeatother



%%%%%%%%%%%%%%%%%%%%%%%%%%%%%%%%%%%%%%%%%
%%  Sprawy metadanych w wynikowym pdf, hyperlinków itp.
%%%%%%%%%%%%%%%%%%%%%%%%%%%%%%%%%%%%%%%%%
% Szablon przygotowano głównie dla pdflatex. Specyficzne komendy dla pdf-owej kompilacj wstawiono 
% w instrukcję warunkową dostarczaną przez pakiet ifpdf 
\usepackage{ifpdf}
%\newif\ifpdf \ifx\pdfoutput\undefined
%\pdffalse % we are not running PDFLaTeX
%\else
%\pdfoutput=1 % we are running PDFLaTeX
%\pdftrue \fi
\ifpdf
 \usepackage{datetime} % INFO: pakiet potrzeby do uzyskania i sformatowania daty 
 \usepackage[pdftex,bookmarks,breaklinks,unicode]{hyperref}
 \usepackage[pdftex]{graphicx}
 \DeclareGraphicsExtensions{.pdf,.jpg,.mps,.png} % po zadeklarowaniu rozszerzeń można będzie wstawiać pliki z grafiką bez konieczności podawania tych rozszerzeń w ich nazwach
\pdfcompresslevel=9
\pdfoutput=1

% Dobrze przygotowany dokument pdf to taki, który zawiera metadane.
% Poniżej zadeklarowano pola metadanych, jakie będą włączone do dokumentu pdf.
% Należy je odpowiednio uzupełnić (domyślnie wstawiany jest tytuł, autor, data modyfikacji)
\makeatletter
\AtBeginDocument{  
  \hypersetup{
	pdfinfo={
    Title = {\@title},
    Author = {\@author},
    Subject={},  
    Keywords={raz, dwa, trzy, cztery}, 
		Producer={}, 
	  CreationDate= {}, % zgodnie ze składnią: {D:yyyymmddhhmmss}, np. D:20210208175600
    ModDate={D:\pdfdate},   % data modyfikacji będzie datą kompilacji
		Creator={pdftex},
	}}
}
\pdftrailerid{} %Remove ID
\pdfsuppressptexinfo15 %Suppress PTEX.Fullbanner and info of imported PDFs

\makeatother
\else             % jeśli kompilacja jest inna niż pdflatex
\usepackage{graphicx}
\DeclareGraphicsExtensions{.eps,.ps,.jpg,.mps,.png}
\fi
\sloppy

\def\UrlBreaks{\do\/\do-\do_} % dodane by lepiej łamać urle 

%\graphicspath{{rys01/}{rys02/}}  % choć można zadeklarować foldery, w jakich pojawiać się mają pliki z grafiką, zaleca się jednak, by tego nie robić


%%%%%%%%%%%%%%%%%%%%%%%%%%%%%%%%%%%%%%%%%
%%  Formatowanie dokumentu
%%%%%%%%%%%%%%%%%%%%%%%%%%%%%%%%%%%%%%%%%
% INFO: Deklaracja głębokościu numeracji
\setcounter{secnumdepth}{2}
\setcounter{tocdepth}{2}
\setsecnumdepth{subsection} 
% INFO: Dodanie kropek po numerach sekcji
\makeatletter
\def\@seccntformat#1{\csname the#1\endcsname.\quad}
\def\numberline#1{\hb@xt@\@tempdima{#1\if&#1&\else.\fi\hfil}}
\makeatother
% INFO: Numeracja rozdziałów i separatory
\renewcommand{\chapternumberline}[1]{#1.\quad}
\renewcommand{\cftchapterdotsep}{\cftdotsep}

% Czcionka do podpisów tabel i rysunków
\captionnamefont{\small}
\captiontitlefont{\small}
% INFO: Makro pozwalające zmienić sposób wypisywania rozdziału (proszę z niego nie korzystać)
%\def\printchaptertitle##1{\fonttitle \space \thechapter.\space ##1} 

% Przedefiniowanie etykiet w podpisach tabel i rysunków
%\AtBeginDocument{% 
        \addto\captionspolish{% 
        \renewcommand{\tablename}{Tab.}% 
}%} 

%\AtBeginDocument{% 
%        \addto\captionspolish{% 
%        \renewcommand{\chaptername}{Rozdział}% 
%}} 

%\AtBeginDocument{% 
        \addto\captionspolish{% 
        \renewcommand{\figurename}{Rys.}% 
}%}


%\AtBeginDocument{% 
        \addto\captionspolish{% 
        \renewcommand{\bibname}{Literatura}% 
}%}

%\AtBeginDocument{% 
        \addto\captionspolish{% 
        \renewcommand{\listfigurename}{Spis rysunków}% 
}%}

%\AtBeginDocument{% 
        \addto\captionspolish{% 
        \renewcommand{\listtablename}{Spis tabel}% 
}%}

%\AtBeginDocument{% 
        \addto\captionspolish

%\AtBeginDocument{% 
    \addto\captionspolish{
\renewcommand\abstractname{Streszczenie} % niepotrzebne, bo przy polskich ustawieniach językowych jest 'Streszczenie'
}%}

%\AtBeginDocument{% 
    \addto\captionsenglish{
\renewcommand\abstractname{Abstract} % niepotrzebne, bo przy polskich ustawieniach językowych jest 'Streszczenie'
}%}

\renewcommand{\abstractnamefont}{\normalfont\Large\bfseries}
\renewcommand{\abstracttextfont}{\normalfont}

\newcommand\mykeywords[1]{\hspace{\absleftindent}\parbox{\linewidth-2.0\absleftindent}{
       \iflanguage{polish}{\textbf{Słowa kluczowe:} #1}{%
			  \iflanguage{english}{\textbf{Keywords:} #1}}{}}
				}

%%%%%%%%%%%%%%%%%%%%%%%%%%%%%%%%%%%%%%%%%%%%%%%%%%%%%%%%%%%%%%%%%%                  
%% Definicje stopek i nagłówków
%%%%%%%%%%%%%%%%%%%%%%%%%%%%%%%%%%%%%%%%%%%%%%%%%%%%%%%%%%%%%%%%%%                  
\addtopsmarks{headings}{%
\nouppercaseheads % added at the beginning
}{%
\createmark{chapter}{both}{shownumber}{}{. \space}
%\createmark{chapter}{left}{shownumber}{}{. \space}
\createmark{section}{right}{shownumber}{}{. \space}
}%use the new settings

\makeatletter
\copypagestyle{outer}{headings}
\makeoddhead{outer}{}{}{\small\itshape\rightmark}
\makeevenhead{outer}{\small\itshape\leftmark}{}{}
\makeoddfoot{outer}{\small\@author:~\@titleShort}{}{\small\thepage}
\makeevenfoot{outer}{\small\thepage}{}{\small\@author:~\@title}
\makeheadrule{outer}{\linewidth}{\normalrulethickness}
\makefootrule{outer}{\linewidth}{\normalrulethickness}{2pt}
\makeatother

% fix plain
\copypagestyle{plain}{headings} % overwrite plain with outer
\makeoddhead{plain}{}{}{} % remove right header
\makeevenhead{plain}{}{}{} % remove left header
\makeevenfoot{plain}{}{}{}
\makeoddfoot{plain}{}{}{}

\copypagestyle{empty}{headings} % overwrite plain with outer
\makeoddhead{empty}{}{}{} % remove right header
\makeevenhead{empty}{}{}{} % remove left header
\makeevenfoot{empty}{}{}{}
\makeoddfoot{empty}{}{}{}


%%%%%%%%%%%%%%%%%%%%%%%%%%%%%%%%%%%%%%%
%% Definicja strony tytułowej 
%%%%%%%%%%%%%%%%%%%%%%%%%%%%%%%%%%%%%%%
\makeatletter
%Uczelnia
\newcommand\uczelnia[1]{\renewcommand\@uczelnia{#1}}
\newcommand\@uczelnia{}
%Wydział
\newcommand\wydzial[1]{\renewcommand\@wydzial{#1}}
\newcommand\@wydzial{}
%Kierunek
\newcommand\kierunek[1]{\renewcommand\@kierunek{#1}}
\newcommand\@kierunek{}
%Specjalność
\newcommand\specjalnosc[1]{\renewcommand\@specjalnosc{#1}}
\newcommand\@specjalnosc{}
%Tytuł po angielsku
\newcommand\titleEN[1]{\renewcommand\@titleEN{#1}}
\newcommand\@titleEN{}
%Tytuł krótki
\newcommand\titleShort[1]{\renewcommand\@titleShort{#1}}
\newcommand\@titleShort{}
%Promotor
\newcommand\promotor[1]{\renewcommand\@promotor{#1}}
\newcommand\@promotor{}

\def\maketitle{%
  \pagestyle{empty}%
%%\garamond 
	\fontfamily{\ebgaramond@family}\selectfont % na stronie tytułowej czcionka garamond
%%%%%%%%%%%%%%%%%%%%%%%%%%%%%%%%%%%%%	
%% Poniżej, w otoczniu picture, wstawiono tytuł i autora. 
%% Tytuł (z autorem) musi znaleźć się w obszarze 
%% odpowiadającym okienku 110mmx75mm, którego lewy górny róg 
%% jest w położeniu 77mm od lewej i 111mm od górnej  krawędzi strony 
%% (tak wynika z wycięcia na okładce). 
%% Poniższy kod musi być użyty dokładnie w miejscu gdzie jest.
%% Jeśli tytuł nie mieści się w okienku, to należy tak pozmieniać 
%% parametry użytych komend, aby ten przydługi tytuł jednak 
%% upakować go do okienka.
%%
%% Sama okładka (kolorowa strona z wycięciem, do pobrania z dydaktyki) 
%% powinna być przycięta o 3mm od każdej z krawędzi.
%% Te 3mm pewnie zostawiono na ewentualne spady czy też specjalną oprawę.
%%%%%%%%%%%%%%%%%%%%%%%%%%%%%%%%%%%%%	
\newlength{\tmpfboxrule}
\setlength{\tmpfboxrule}{\fboxrule}
\setlength{\fboxsep}{2mm}
\setlength{\fboxrule}{0mm} 
%\setlength{\fboxrule}{0.1mm} %% INFO: Jeśli chcemy zobaczyć ramkę, wystarczy odmarkować tę linijkę
\setlength{\unitlength}{1mm}
\begin{picture}(0,0)
\put(26,-124){\fbox{
\parbox[c][71mm][c]{104mm}{\centering%\lineskip=34pt 
\fontsize{16pt}{18pt}\selectfont \@title\\[5mm]
\fontsize{16pt}{18pt}\selectfont \@titleEN\\[20mm]
\fontsize{16pt}{18pt}\selectfont AUTOR:\\[2mm]
\fontsize{14pt}{16pt}\selectfont \@author}
}
}
\end{picture}
\setlength{\fboxrule}{\tmpfboxrule} 
%%%%%%%%%%%%%%%%%%%%%%%%%%%%%%%%%%%%%
%% Reszta strony z nazwą uczelni, wydziału, kierunkiem, specjalnością
%% promotorem, oceną pracy, miastem i rokiem
	{\centering%\vspace{-1cm}
		{\fontsize{22pt}{24pt}\selectfont \@uczelnia}\\[0.4cm]
		{\fontsize{22pt}{24pt}\selectfont \@wydzial}\\[0.5cm]
		  \hrule %\vspace*{0.7cm}
	}
{\flushleft\fontsize{14pt}{16pt}\selectfont%
\begin{tabular}{ll}
KIERUNEK: & \@kierunek\\
SPECJALNOŚĆ: & \@specjalnosc\\
\end{tabular}\\[1.3cm]
}
{\centering
{\fontsize{32pt}{36pt}\selectfont PRACA DYPLOMOWA}\\[0.5cm]
{\fontsize{32pt}{36pt}\selectfont MAGISTERSKA}\\[2.5cm]
}
\vfill
\begin{tabularx}{\linewidth}{p{6cm}X}
		&{\fontsize{16pt}{18pt}\selectfont PROWADZĄCY PRACĘ:}\\[2mm] %INFO: tutaj jest miejsce na nazwisko promotora pracy
		&{\fontsize{14pt}{16pt}\selectfont \@promotor}\\[10mm]
%		&{\fontsize{16pt}{18pt}\selectfont OCENA PRACY:}\\[20mm]
% Tekst 'OCENA PRACY:' można usunąć, jeśli praca ma być dostarczona bez podpisu promotora (w dobie pandemii)
% Czyli zamiast powyżej zdefiniowanego wiersza tabeli można wpisać:
   &{\fontsize{16pt}{18pt}\selectfont }\\[20mm]
	\end{tabularx}
\vspace{2cm}
\hrule\vspace*{0.3cm}
{\centering
{\fontsize{16pt}{18pt}\selectfont \@date}\\[0cm]
}
%\ungaramond
\normalfont
 \cleardoublepage
}
\makeatother
%%%%%%%%%%%%%%%%%%%%%%%%%%%%%%%%%%%%%%%%%

%\AtBeginDocument{\addtocontents{toc}{\protect\thispagestyle{empty}}}


%%%%%%%%%%%%%%%%%%%%%%%%%%%%%%%%%%%%%%%%%
%%  Metadane dokumentu 
%%%%%%%%%%%%%%%%%%%%%%%%%%%%%%%%%%%%%%%%%
\title{Metoda automatycznej oceny podobieństwa semantycznego tekstów na potrzeby analizy dyskursu publicznego}
\titleShort{Metoda automatycznej oceny podobieństwa semantycznego tekstów \dots}
\titleEN{Automatic assessment of semantic similarity of texts for the analysis of public discourse}
\author{Marcin Kotas}
\uczelnia{POLITECHNIKA WROCŁAWSKA}
\wydzial{WYDZIAŁ ELEKTRONIKI}
\kierunek{INFORMATYKA (INF)}
\specjalnosc{GRAFIKA I SYSTEMY MULTIMEDIALNE (IGM)}
\promotor{dr inż. Tomasz Walkowiak K30W04D03}
\date{WROCŁAW, 2021}

% Ustawienie odstępu od góry w nienumerowanych rozdziałach oraz wykazach:
% Spis treści, Spis tabel, Spis rysunków, Indeks rzeczowy

%\newlength{\linespace}
%\setlength{\linespace}{-\beforechapskip-\topskip+\headheight+\topsep}
%%%\makechapterstyle{noNumbered}{%
%%%\renewcommand\chapterheadstart{\vspace*{\linespace}}
%%%}

%% powyższa komenda załatwia to, co robią komendy poniższe dla spisów
%\renewcommand*{\tocheadstart}{\vspace*{\linespace}}
%\renewcommand*{\lotheadstart}{\vspace*{\linespace}}
%\renewcommand*{\lofheadstart}{\vspace*{\linespace}}

%%%%%%%%%%%%%%%%%%%%%%%%%%%%%%%%%%%%%%%%%
%                  Początek dokumentu 
%%%%%%%%%%%%%%%%%%%%%%%%%%%%%%%%%%%%%%%%%
% INFO: Za pomocą polecenia \includeonly{} można dokonać selekcji części składowych dokumentu (plików z latexowym kodem), jakie mają być kompilowane. 
%       Przydaje się to szczególnie podczas pracy nad dużymi dokumentami. 
%       Bo im mniej jest części składowych do przetworzenia, tym szybsza jest kompilacja.
%       Proszę nie mylić tej komendy z poleceniem \include{}, którą używa się do zadeklarowania części składowych dokumentu (plików z latexowym kodem).

%\includeonly{skroty,rozdzial01}  

\usepackage{todonotes}

\begin{document}
% Komendami poniżej można przełączyć odstęp między liniami. Proszę jednak tego nie robić !!!
%\SingleSpacing
%\OnehalfSpacing
%\DoubleSpacing

%\settypeoutlayoutunit{cm} % do debugowania
%\typeoutstandardlayout    % wypisuje na stdout informacje o ustawieniach

%\frontmatter
\pdfbookmark[0]{Tytuł}{Tytul.1}
\maketitle
\clearpage
%\chapterstyle{noNumbered}
\pagestyle{outer}
\pdfbookmark[0]{Streszczenie}{streszczenie.1}
%\phantomsection
%\addcontentsline{toc}{chapter}{Streszczenie}
%%% Poniższe zostało niewykorzystane (tj. zrezygnowano z utworzenia nienumerowanego rozdziału na abstrakt)
%%%\begingroup
%%%\setlength\beforechapskip{48pt} % z jakiegoś powodu była maleńka różnica w położeniu nagłówka rozdziału numerowanego i nienumerowanego
%%%\chapter*{\centering Abstrakt}
%%%\endgroup
%%%\label{sec:abstrakt}
%%%Lorem ipsum dolor sit amet eleifend et, congue arcu. Morbi tellus sit amet, massa. Vivamus est id risus. Sed sit amet, libero. Aenean ac ipsum. Mauris vel lectus. 
%%%
%%%Nam id nulla a adipiscing tortor, dictum ut, lobortis urna. Donec non dui. Cras tempus orci ipsum, molestie quis, lacinia varius nunc, rhoncus purus, consectetuer congue risus. 
%\mbox{}\vspace{2cm} % można przesunąć, w zależności od długości streszczenia
\begin{abstract}
	W ostatnich latach powstały przełomowe rozwiązania z dziedziny przetwarzania języka, oparte na modelach językowych bazujących na Transformerach.
	W pracy zbadano skuteczność takich modeli dla języka polskiego w kontekście problemu modelowania tematycznego.
	Algorytmy porównano z wiodącą metodą LDA\@.
	Testy przeprowadzono na Korpusie Dyskursu Parlamentarnego, który zawiera różnorodne i długie wypowiedzi.
	Modelowanie tematyczne oparto na algorytmie BERTopic, który składa się z następujących kroków:
		osadzenie wektorów zdań, redukcja wymiarowości wektorów i klastrowanie w grupy tematyczne poprzez ocenę podobieństwa semantycznego.
	
		Z powodzeniem utworzono modele przewyższające jakością LDA, choć nie wszystkie pierwotne założenia okazały się słuszne.
	Jak pokazała analiza wyników, dokładność zaawansowanych modeli takich jak \emph{Sentence-Transformers} czy \emph{Universal Sentence Encoder}
		maleje dla długich wypowiedzi.
	Okazuje się, że analizując tak długie teksty, lepiej jest skorzystać ze zwykłych metod statystycznych \emph{TF-IDF} poprzedzonych lematyzacją danych wejściowych.
	Wektory wygenerowane w taki sposób efektywniej wykorzystują duży rozmiar dokumentów,
		podczas gdy pozostałe modele osiągają gorszą dokładność wynikającą z uśredniania wektorów zdań bądź analizy tylko początkowych fragmentów wypowiedzi.
\end{abstract}
\mykeywords{nlp, bert, bertopic, lda, modelowanie tematyczne}\\ 
% Dobrze byłoby skopiować słowa kluczowe do metadanych dokumentu pdf (w pliku Dyplom.tex)
% Niestety, zaimplementowane makro nie robi tego z automatu, więc pozostaje kopiowanie ręczne.

{
\selectlanguage{english}
\begin{abstract}
	Recent advancements in Natural Language Processing produced state of the art language models based on Transformers.
	This study aims to determine, how well these models perform for topic modeling task in polish language.
	With LDA being the most commonly used method, it is assumed as the baseline.
	The tests have been carried out on the Polish Parliamentary Corpus, which consists of diverse and long utterances.
	Topic modeling with is based on BERTopic, which consists of the following steps:
		sentence embedding, vector dimensionality reduction and clustering into topic groups by comparing semantic similarity.
	
		It has shown promising results far exceeding those achieved by LDA,
		however embedding lemmatized sentences with statistical \emph{TF-IDF} method has proven to be most efficient for such long sequences.
	Other methods (\emph{Sentence-Transformers}, \emph{Universal Sentence Encoder}) lose their accuracy with document length increase.
	This is caused by limited input length when analysing whole utterance, or information loss in averaging vectors when splitting document into sentences.
\end{abstract}
\mykeywords{nlp, bert, bertopic, lda, topic modeling}
}
 
\clearpage
\pdfbookmark[0]{Spis treści}{spisTresci.1}%
%%\phantomsection
%%\addcontentsline{toc}{chapter}{Spis treści}
\tableofcontents* 
\clearpage
\pdfbookmark[0]{Spis rysunków}{spisRysunkow.1} % jeśli chcemy mieć w spisie treści, to zamarkować tę linię, a odmarkować linie poniższe
%%\phantomsection
%%\addcontentsline{toc}{chapter}{Spis rysunków}
\listoffigures*
\clearpage
\pdfbookmark[0]{Spis tabel}{spisTabel.1} %
%%\phantomsection
%%\addcontentsline{toc}{chapter}{Spis tabel}
\listoftables*
\clearpage
\pdfbookmark[0]{Spis listingów}{spisListingow.1} %
%%\phantomsection
%%\addcontentsline{toc}{chapter}{Spis listingów}
\lstlistoflistings*
\clearpage
\pdfbookmark[0]{Skróty}{skroty.1}% 
%%\phantomsection
%%\addcontentsline{toc}{chapter}{Skróty}
\chapter*{Skróty}\label{sec:skroty}
\noindent\vspace{-\topsep-\partopsep-\parsep} % Jeśli zaczyna się od otoczenia description, to otoczenie to ląduje lekko niżej niż wylądowałby zwykły tekst, dlatego wstawiano przesunięcie w pionie
\begin{description}[labelwidth=*]
  \item [PPC] (ang.\ \emph{Polish Parliamentary Corpus}) %-- jednostka arytmetyczno--logiczna
  \item [NLP] (ang.\ \emph{Natural Language Processing})
\end{description}
 %skróty można sobie pominąć


%\mainmatter
\chapterstyle{default}
% !TeX root = ./Dyplom.tex

\chapter{Wstęp}
 
\section{Cel i zakres pracy}
	Celem pracy jest zbadanie skuteczności nowoczesnych metod z dziedziny przetwarzania języka naturalnego
		w kontekście metod oceny podobieństwa semantycznego na potrzeby analizy dyskursu publicznego.
	Prace przeprowadzone zostaną na zbiorze wypowiedzi posłów Sejmu III Rzeczpospolitej Polskiej pozyskanym z Korpusu Dyskursu Parlamentarnego.
	Ze względu na brak istniejących oznaczeń dokumentów, postanowiono skupić się na problemie z dziedziny uczenia nienadzorowanego, jakim jest modelowanie tematyczne.
	Jest to zadanie, w którym model językowy grupuje podobne dokumenty poprzez analizę semantyczną tworząc zbiory reprezentujące spójne tematy.

	Z wybranym zbiorem dokumentów wiążą się dodatkowe wyzwania, mianowicie nie wszystkie wypowiedzi zawierają wyróżniające słowa
		(np.\ kiedy wypowiedź jest odpowiedzią na zapytanie to osoba nie zawsze zawrze w wypowiedzi kontekst).
	Dodatkowo, większość współczesnych metod NLP operuje na krótkich dokumentach, podczas gdy wypowiedzi sejmowe mają formę nierzadko długich przemówień.
	Te aspekty korpusu oraz wysokie zróżnicowanie tematyczne kwestii poruszanych w sejmie sprawiają,
		że jest to intrygujący i obiecujący zestaw dokumentów do analizy porównawczej.

	Modelowanie tematyczne z wykorzystaniem wektorów kodujących dokumenty odbywa się w czterech krokach:
	\begin{enumerate}
		\item osadzenie wektorów dokumentów,
		\item redukcja wymiarowości wektorów,
		\item grupowanie na podstawie podobieństwa,
		\item utworzenie reprezentacji słownej każdego tematu (zbioru słów najlepiej opisujących zgrupowane wypowiedzi).
	\end{enumerate}
	Krok pierwszy może zostać wykonany każdą metodą generującą wektory kodujące dokumenty na jednej wspólnej przestrzeni.
	W ostatnich latach powstały przełomowe rozwiązania oparte na modelach językowych bazujących na architekturze BERT\@.
	W pracy przeprowadzone zostaną badania mające na celu ustalenie, jak skuteczne są takie modele w kontekście analizowanego problemu.
	Dodatkowo, analiza przeprowadzona zostanie również na modelach wykorzystujących alternatywne podejścia kodowania dokumentów,
		co daje sumarycznie trzy następujące metody:
	\begin{itemize}
		\item architektura BERT,
		\item sieć konwolucyjna,
		\item TF-IDF (metoda statystyczna).
	\end{itemize}
	Ponadto, testy wykonane zostaną dla różnych kombinacji parametrów wykorzystywanych algorytmów do redukcji wymiarowości i grupowania wektorów,
		aby odnaleźć optymalne wartości dla analizowanego korpusu.

	Jako alternatywną metodę modelowania tematycznego dokumentów, która stanowi punkt odniesienia,
		przeprowadzona zostanie analiza modelem LDA\@.
	Jest to obecnie jedno z najpopularniejszych narzędzi przeprowadzania modelowania tematycznego\cite{LDA_popularity},
		które bazuje na metodach statystycznych.

\section{Układ pracy}
	W kolejnym rozdziale omówiony zostanie wykorzystywany korpus wypowiedzi sejmowych,
		oraz działania jakie zostały wykonane podczas jego przetwarzania.
	W rozdziale trzecim przedstawiono wszystkie analizowane metody generowania wektorów dokumentów.
	Rozdział czwarty zawiera opisy kolejnych kroków wykonywanych w ramach modelowania tematycznego
		--- redukcja wymiarowości wektorów, grupowanie oraz generowanie reprezentacji tematu.
	W kolejnym rozdziale przedstawione zostały sposoby ewaluacji tematów odnalezionych w korpusie.
	W szóstym rozdziale opisano uzyskane wyniki zarówno pod kątem wykorzystywanych metod i parametrów, jak i zastosowanych metryk.
	W ostatnim rozdziale zawarto podsumowanie pracy.
\chapter{Praca z szablonem}
\section{Struktura projektu}
Pisząc pracę w systemie LaTeX zwykle przyjmuje się jakąś konwencję co do nazewnictwa tworzonych plików, ich położenia oraz powiązań. Przygotowując niniejszy szablon założono, że projekt będzie się składał z pliku głównego, plików z kodem kolejnych rozdziałów i dodatków (włączanych do kompilacji w dokumencie głównym), katalogów z plikami grafik (o nazwach wskazujących na rozdziały, w których grafiki te zostaną wstawione), pliku ze skrótami (opcjonalny), pliku z danymi bibliograficznymi (plik \texttt{dokumentacja.bib}). Taki ,,układ'' zapewnia porządek oraz pozwala na selektywną kompilację rozdziałów. 

Przyjętą konwencję da się opisać jak następuje:
\begin{itemize}
\item Plikiem głównym jest plik \texttt{Dyplom.tex}. To w nim znajdują się deklaracje wszystkich używanych styli, definicje makr oraz ustawień, jak również polecenie \verb+\begin{document}+. 
\item Teksty redagowane są w osobnych plikach. Pliki te zamieszczone są w katalogu głównym (tym samym, co plik \texttt{Dyplom.tex}).
\item W pliku \texttt{streszczenie.tex} powinien pojawić się tekst streszczenia ze słowami kluczowymi (tekst ten oraz słowa kluczowe będzie można wykorzystać do wypełnienia formularzy pojawiających się podczas wysyłania pracy do analizy antyplagiatowej w systemie ASAP)
\item Plik \texttt{skroty.tex} powinien zawierać wykaz użytych skrótów. Jeśli w pracy nie stosuje się skrótów czy akronimów lub ich liczba jest nieznaczna, wtedy należy zrezygnować z tego pliku. 
\item Tekst kolejnych rozdziałów powinien pojawić się w plikach o nazwach zawierających numery tych rozdziałów. Według przyjętej konwencji \texttt{rozdzial01.tex} to plik pierwszego rozdziału (ze Wstępem), \texttt{rozdzial02.tex} to plik z treścią drugiego rozdziału itd. 
\item Teksty dodatków mają być redagowane w osobnych plikach o nazwach zawierających literę dodatku. Pliki te, podobnie do plików z tekstem rozdziałów, zamieszczane są w katalogu głównym. I~tak \texttt{dodatekA.tex} oraz \texttt{dodatekB.tex} to, odpowiednio, pliki z treścią dodatku A oraz dodatku B.
\item Każdemu rozdziałowi i dodatkowi towarzyszy katalog przeznaczony do składowania dołączanych w nim grafik. I tak \texttt{rys01} to katalog na pliki z grafikami dołączanymi do rozdziału pierwszego, \texttt{rys02} to katalog na pliki z grafikami dołączanymi do rozdziału drugiego itd.
Podobnie \texttt{rysA} to katalog na pliki z grafikami dołączanymi w dodatku A itd.
\item W katalogu głównym zamieszczany jest plik \texttt{dokumentacja.bib} zawierający bazę danych bibliograficznych.
\item Jeśli praca nad dokumentem odbywa się w \texttt{TeXnicCenter}, to środowisko to wymusza konieczność podania pliku projektu (plik projektu to coś na styl pliku z definicją solucji). Plikiem projektu, który umieszczono w szablonie jest \texttt{Dyplom.tcp}. Generalnie plik projektu nie jest wymagany do latexowej kompilacji. Niemniej pozwala zapamiętać ustawienia środowiska (w tym ustawienia językowe potrzebne do sprawdzania poprawności wyrazów -- patrz następny podrozdział).  
\end{itemize}

\begin{table}[htb]
\centering\small
\caption{Pliki źródłowe szablonu oraz wyniki kompilacji}
\label{tab:szablon}
\begin{tabularx}{\linewidth}{|p{.55\linewidth}|X|}\hline
Źródła & Wyniki kompilacji \\ \hline\hline
\verb?Dokument.tex? - dokument główny\newline
\verb?Dokument.tcp? -- plik projektu TeXnicCenter (opcjonalny)\newline
\verb?streszczenie.tex? -- plik streszczenia\newline
\verb?skroty.tex? -- plik ze skrótami\newline
\verb?rozdzial01.tex? -- plik rozdziału \texttt{01}\newline
\verb?...?\newline
\verb?dodatekA.tex? -- plik dodatku \texttt{A}\newline
\verb?...?\newline
\verb?rys01? -- katalog na rysunki do rozdziału \texttt{01}\newline
\verb?   |- fig01.png? -- plik grafiki\newline
\verb?   |- ...?\newline
\verb?...?\newline
\verb?rysA? -- katalog na rysunki do dodatku \texttt{A}\newline
\verb?   |- fig01.png? -- plik grafiki\newline
\verb?   |- ...?\newline
\verb?...?\newline
\verb?dokumentacja.bib? -- plik danych bibliograficznych\newline
\verb?Dyplom.ist? -- plik ze stylem indeksu\newline
\verb?by-nc-sa.png? -- plik z ikonami CC\newline
 &
\verb?Dyplom.bbl?\newline
\verb?Dyplom.blg?\newline
\verb?Dyplom.ind?\newline
\verb?Dyplom.idx?\newline
\verb?Dyplom.lof?\newline
\verb?Dyplom.log?\newline
\verb?Dyplom.lot?\newline
\verb?Dyplom.out?\newline
\verb?Dyplom.pdf? -- dokument wynikowy\newline
\verb?Dyplom.syntex?\newline
\verb?Dyplom.toc?\newline
\verb?Dyplom.tps?\newline
\verb?*.aux?\newline 
\verb?Dyplom.synctex?\newline\\
\hline
\end{tabularx}
\end{table}

\section{Kodowanie znaków}
System LaTeX obsługuje wielojęzyczność. Można tworzyć w nim dokumenty z tekstem zawierającym różne znaki diakrytyczne. 
Należy jednak zdawać sobie sprawę, w jaki sposób znaki te są obsługiwane. Otóż na kodowanie znaków należy patrzeć z dwóch perspektyw: perspektywy edytowania kodu latexowego oraz perspektywy kodowania dokumentu wynikowego i użytych czcionek.

Kod latexowy może być edytowany w dowolnym tekstowym edytorze. Zastosowane kodowanie znaków w tym edytorze musi być znane latexowi, inaczej kompilacja tego kodu się nie powiedzie. Informację o tym kodowaniu przekazuje się w opcjach pakietu \texttt{inputenc}. Niniejszy szablon przygotowano w systemie Windows, a latexowe źródła umieszczono w plikach ANSI z użyciem strony kodowej cp1250. Dlatego w poleceniu \verb+\usepackage[cp1250]{inputenc}+ jako opcję wpisano \texttt{cp1250}.


Szablon można wykorzystać również przy innych kodowaniach i w innych systemach. Jednak wtedy konieczna będzie korekta dokumentu \texttt{Dyplom.tex} odpowiednio do wybranego przypadku. Korekta ta polegać ma na zamianie polecenia \verb+\usepackage[cp1250]{inputenc}+  na polecenie \verb+\usepackage[utf8]{inputenc}+ oraz konwersji znaków i zmiany kodowania istniejących plików ze źródłem latexowego kodu (plików o rozszerzeniu \texttt{*.tex} oraz \texttt{*.bib}).

Kodowanie znaków jest istotne również przy edytowaniu bazy danych bibliograficznych (pliku \texttt{dokumentacja.bib}). Aby \texttt{bibtex} poprawnie interpretował polskie znaki plik \texttt{dokumentacja.bib} powinien być zakodowany w ANSI, CR+LF (dla ustawień jak w szablonie). 
W szczególności, jeśli ominąć chce się problem kodowania, polskie znaki w bazie danych bibliograficznych można zastąpić odpowiednią notacją: \verb|\k{a}| \verb|\'c| \verb|\k{e}| \verb|\l{}| \verb|\'n| \verb|\'o| \verb|\'s| \verb|\'z| \verb|\.z| \verb|\k{A}| \verb|\'C| \verb|\k{E}| \verb|\L{}| \verb|\'N| \verb|\'O| \verb|\'S| \verb|\'Z| \verb|\.Z|. 

Samo kodowanie plików może być źródłem paru problemów. Chodzi o to, że użytkownicy pracujący z edytorami tekstów pod linuxem mogą generować pliki zakodowane w UTF-8 bez BOM (lub z BOM -- co nie jest zalecane), a pod windowsem -- pliki ANSI ze znakami ze strony kodewej \texttt{cp1250}. A z takimi plikami różne edytory różnie sobie radzą. W szczególności edytor TeXnicCenter podczas otwierania plików może potraktować jego zawartość jako UTF8 lub ANSI -- prawdopodobnie interpretuje z jakim kodowaniem ma do czynienia na podstawie obecności w pliku znaków specjalnych. Bywa, że choć wszystko w TeXnicCenter wygląda OK, to jednak kompilacja latexowa ,,nie idzie''. Problemem mogą być właśnie pierwsze bajty, których nie widać w edytorze. 

Do konwersji kodowania można użyć Notepad++ (jest tam opcja ,,konwertuj'' - nie mylić z opcją ,,koduj'', która przekodowuje znaki, jednak nie zmienia sposobu kodowania pliku).

Jeśli chodzi o drugą perspektywę, tj.\ kodowanie znaków w dokumencie wynikowym, to sprawa jest bardziej skomplikowana. Wiąże się z nią zarządzanie czcionkami, definiowanie mapowania itp. Szablon przygotowano tak, by wynikowy dokument zawierał polskie znaki diakrytyczne, które nie są zlepkami literki i ogonka.

\section{Kompilacja szablonu}
Kompilację szablonu można uruchamiać na kilka różnych sposobów. Wszystko zależy od używanego systemu operacyjnego, zainstalowanej na nim dystrybucji latexa oraz dostępnych narzędzi. Zazwyczaj kompilację rozpoczyna się wydając polecenie z linii komend lub uruchamia się ją za pomocą narzędzi zintegrowanych środowisk.

Kompilacja z linii komend polega na uruchomieniu w katalogu, w którym rozpakowano źródła szablonu, następującego polecenia:
\begin{lstlisting}[basicstyle=\ttfamily]
> pdflatex Dyplom.tex
\end{lstlisting}
gdzie \texttt{pdflatex} to nazwa kompilatora, zaś \texttt{Dyplom.tex} to nazwa głównego pliku redagowanej pracy. 
W przypadku korzystania ze środowiska \texttt{TeXnicCenter} należy otworzyć dostarczony w szablonie plik projektu \texttt{Dyplom.tcp}, a następnie uruchomić kompilację narzędziami dostępnymi w pasku narzędziowym.

Aby poprawnie wygenerowały się wszystkie referencje (spis treści, odwołania do tabel, rysunków, pozycji literaturowych, równań itd.) kompilację \texttt{pdflatex} należy wykonać dwukrotnie, a~czasem nawet trzykrotnie. Wynika to z konieczności zapamiętywania wyników kompilacji i ich wykorzystywania w kolejnych przebiegach. Tak dzieje się przy generowaniu odwołań do pozycji literaturowych oraz tworzeniu wykazu literatury). 

Wygenerowanie danych bibliograficznych zapewnia kompilacja \texttt{bibtex} uruchamiana po kompilacji \texttt{pdfltex}. Można to zrobić z linii komend:
\begin{lstlisting}[basicstyle=\ttfamily]
> bibtex Dyplom
\end{lstlisting}
lub wybierając odpowiednią pozycję z paska narzędziowego wykorzystywanego środowiska. Po kompilacji za pomocą \texttt{bibtex} na dysku pojawi się plik \texttt{Dyplom.bbl}. Dopiero po kolejnych dwóch kompilacjach \texttt{pdflatex} dane z tego pliku zostaną odpowiednio przetworzone i zrenderowane w wygenerowanym dokumencie. Tak więc po każdym wstawieniu nowego cytowania w kodzie dokumentu uzyskanie poprawnego formatowania dokumentu wynikowego wymaga powtórzenia następującej sekwencji kroków kompilacji:
\begin{lstlisting}[basicstyle=\ttfamily]
> pdflatex Document.tex
> bibtex Document
> latex Document.tex
> latex Document.tex
\end{lstlisting}
Szczegóły dotyczące przygotowania danych bibliograficznych oraz zastosowania cytowań przedstawiono w podrozdziale \ref{sec:literatura}.

W głównym pliku zamieszczono polecenia pozwalające sterować procesem kompilacji poprzez włączanie bądź wyłączanie kodu źródłowego poszczególnych rozdziałów. Włączanie kodu do kompilacji zapewniają instrukcje \verb+\include+ oraz \verb+\includeonly+. Pierwsza z nich pozwala włączyć do kompilacji kod wskazanego pliku (np.\ kodu źródłowego pierwszego rozdziału \verb+% !TeX root = ./Dyplom.tex

\chapter{Wstęp}
 
\section{Cel i zakres pracy}
	Celem pracy jest zbadanie skuteczności nowoczesnych metod z dziedziny przetwarzania języka naturalnego
		w kontekście metod oceny podobieństwa semantycznego na potrzeby analizy dyskursu publicznego.
	Prace przeprowadzone zostaną na zbiorze wypowiedzi posłów Sejmu III Rzeczpospolitej Polskiej pozyskanym z Korpusu Dyskursu Parlamentarnego.
	Ze względu na brak istniejących oznaczeń dokumentów, postanowiono skupić się na problemie z dziedziny uczenia nienadzorowanego, jakim jest modelowanie tematyczne.
	Jest to zadanie, w którym model językowy grupuje podobne dokumenty poprzez analizę semantyczną tworząc zbiory reprezentujące spójne tematy.

	Z wybranym zbiorem dokumentów wiążą się dodatkowe wyzwania, mianowicie nie wszystkie wypowiedzi zawierają wyróżniające słowa
		(np.\ kiedy wypowiedź jest odpowiedzią na zapytanie to osoba nie zawsze zawrze w wypowiedzi kontekst).
	Dodatkowo, większość współczesnych metod NLP operuje na krótkich dokumentach, podczas gdy wypowiedzi sejmowe mają formę nierzadko długich przemówień.
	Te aspekty korpusu oraz wysokie zróżnicowanie tematyczne kwestii poruszanych w sejmie sprawiają,
		że jest to intrygujący i obiecujący zestaw dokumentów do analizy porównawczej.

	Modelowanie tematyczne z wykorzystaniem wektorów kodujących dokumenty odbywa się w czterech krokach:
	\begin{enumerate}
		\item osadzenie wektorów dokumentów,
		\item redukcja wymiarowości wektorów,
		\item grupowanie na podstawie podobieństwa,
		\item utworzenie reprezentacji słownej każdego tematu (zbioru słów najlepiej opisujących zgrupowane wypowiedzi).
	\end{enumerate}
	Krok pierwszy może zostać wykonany każdą metodą generującą wektory kodujące dokumenty na jednej wspólnej przestrzeni.
	W ostatnich latach powstały przełomowe rozwiązania oparte na modelach językowych bazujących na architekturze BERT\@.
	W pracy przeprowadzone zostaną badania mające na celu ustalenie, jak skuteczne są takie modele w kontekście analizowanego problemu.
	Dodatkowo, analiza przeprowadzona zostanie również na modelach wykorzystujących alternatywne podejścia kodowania dokumentów,
		co daje sumarycznie trzy następujące metody:
	\begin{itemize}
		\item architektura BERT,
		\item sieć konwolucyjna,
		\item TF-IDF (metoda statystyczna).
	\end{itemize}
	Ponadto, testy wykonane zostaną dla różnych kombinacji parametrów wykorzystywanych algorytmów do redukcji wymiarowości i grupowania wektorów,
		aby odnaleźć optymalne wartości dla analizowanego korpusu.

	Jako alternatywną metodę modelowania tematycznego dokumentów, która stanowi punkt odniesienia,
		przeprowadzona zostanie analiza modelem LDA\@.
	Jest to obecnie jedno z najpopularniejszych narzędzi przeprowadzania modelowania tematycznego\cite{LDA_popularity},
		które bazuje na metodach statystycznych.

\section{Układ pracy}
	W kolejnym rozdziale omówiony zostanie wykorzystywany korpus wypowiedzi sejmowych,
		oraz działania jakie zostały wykonane podczas jego przetwarzania.
	W rozdziale trzecim przedstawiono wszystkie analizowane metody generowania wektorów dokumentów.
	Rozdział czwarty zawiera opisy kolejnych kroków wykonywanych w ramach modelowania tematycznego
		--- redukcja wymiarowości wektorów, grupowanie oraz generowanie reprezentacji tematu.
	W kolejnym rozdziale przedstawione zostały sposoby ewaluacji tematów odnalezionych w korpusie.
	W szóstym rozdziale opisano uzyskane wyniki zarówno pod kątem wykorzystywanych metod i parametrów, jak i zastosowanych metryk.
	W ostatnim rozdziale zawarto podsumowanie pracy.+). Druga, jeśli zostanie zastosowana, pozwala określić, które z~plików zostaną skompilowane w całości (na przykład kod źródłowy pierwszego i drugiego rozdziału \verb+\includeonly{rozdzial01.tex,rozdzial02.tex}+). Brak nazwy pliku na liście w poleceniu \verb+\includeonly+ przy jednoczesnym wystąpieniu jego nazwy w poleceniu \verb+\include+ oznacza, że w kompilacji zostaną uwzględnione referencje wygenerowane dla tego pliku wcześniej, sam zaś kod źródłowy pliku nie będzie kompilowany. 

W szablonie wykorzystano klasę dokumentu \texttt{memoir} oraz wybrane pakiety. Podczas kompilacji szablonu w \texttt{MikTeXu} wszelkie potrzebne pakiety zostaną zainstalowane automatycznie (jeśli \texttt{MikTeX} zainstalowano z opcją dynamicznej instalacji brakujących pakietów). W przypadku innych dystrybucji latexowych może okazać się, że pakiety te trzeba doinstalować ręcznie (np.\ pod linuxem z \texttt{TeXLive} trzeba doinstalować dodatkową zbiorczą paczkę, a jeśli ma się menadżera pakietów latexowych, to pakiety latexowe można instalować indywidualnie).

Jeśli w szablonie będzie wykorzystany indeks rzeczowy, kompilację źródeł trzeba będzie rozszerzyć o kroki potrzebne na wygenerowanie plików pośrednich \texttt{Dokument.idx} oraz \texttt{Dokument.ind} oraz dołączenia ich do finalnego dokumentu (podobnie jak to ma miejsce przy generowaniu wykazu literatury).
Szczegóły dotyczące generowania indeksu rzeczowego opisano w podrozdziale~\ref{sec:indeks}.

\section{Sprawdzanie poprawności tekstu}
Większość środowisk ułatwiających pisanie latexowych dokumentów wspiera sprawdzenie poprawności tekstu (ang.~\emph{spell checking}). Wystarczy odpowiednio je skonfigurować. Niestety, proponowana przez narzędzia korekta nie jest genialna. Bazuje ona na prostym porównywaniu wyrazów (z końcówkami). Nie wbudowano w nią żadnej większej inteligencji. Tak więc proszę nie porównywać jej z korektą oferowaną w narzędziach MS Office (tam jest ona dużo bardziej zaawansowana).

TeXnicCenter korzysta ze słowników do pobrania ze strony openoffice (\url{https://extensions.openoffice.org/}).
Aby sprawę uprościć słowniki dla języka polskiego (pliki \texttt{pl\_PL.aff} oraz \texttt{pl\_PL.dic}) dołączono do szablonu (są w katalogu \texttt{Dictionaries}). Pliki te należy umieścić w katalogu \texttt{C:\\Program Files\\TeXnicCenter\\Dictionaries}, a w konfiguracji projektu (Tools/Options/Spelling należy wybrać \texttt{Language: pl}, \texttt{Dialect: PL}. Jeśli główny tekst pracy pisany jest w innym języku, to trzeba zmienić słownik.

Zaskakujące może jest to, że \emph{spell checker} w TeXnicCenter działa zarówno przy pracy na plikach UTF-8, jak i na plikach ANSI. Jeśli byłyby jakieś problemy ze słownikiem wynikające z kodowania znaków, wtedy słownik trzeba przekodować. To powinno pomóc.

\section{Wersjonowanie}
W trakcie edytowania pracy w systemie latex dobrą praktyką jest wersjonowanie tworzonego kodu. 
Do wersjonowania zaleca się wykorzystać system git. Opis sposobu pracy z tym systemem opisano w licznych tutorialach dostępnych w sieci. Szczególnie godnym polecenia zasobem jest strona domowa projektu \url{https://git-scm.com/}.

Zwykle pracę z gitem rozpoczyna się od utworzenia repozytorium zdalnego i lokalnego. Lokalne służy do bieżącej pracy, zdalne -- do współpracy z innymi użytkownikami (z promotorem). 

Po utworzeniu repozytorium lokalnego i jego gałęzi (czy będzie to master czy inna gałąź -- wszystko zależy od ustaleń między zainteresowanymi) należy skopiować do niego wszystkie pliki dostarczone w szablonie, a po ich wstępnym przeredagowaniu należy je zaznaczyć do wersjonowania. Potem należy wysłać zmiany na repozytorium zdalne (możliwa jest też ścieżka odwrotna - można zacząć od zmian na repozytorium zdalnym, które pobrane będą do repozytorium lokanego).

Podczas kompilowania projektu będą powstawały pliki pomocnicze. Plików tych nie należy wersjonować (zabierają niepotrzebnie miejsce, a przecież zawsze można je odtworzyć uruchamiając kompilację na źródłach). Git posiada mechanizm automatycznego odrzucania plików niepodlegających wersjonowaniu. Mechanizm ten bazuje na wykorzystaniu pliku konfiguracyjnego \texttt{.gitignore} zamieszczonego w katalogu głównym repozytorium. O szczegółach \texttt{.gitignore} można poczytać  na stronie \url{https://git-scm.com/docs/gitignore}. 

W sieci można znaleźć liczne propozycje plików konfiguracyjnych \texttt{.gitignore} dopasowanych do potrzeb latexowej kompilacji. Nie trzeba ich jednak szukać. Aby sprawę uprościć w szablonie zamieszczono specjalnie spreparowany taki plik. Zawiera on, między innymi, wpis mówiący o tym, by nie wersjonować pliku wynikowego \texttt{Dyplom.pdf}. Plik ten należy umieścić w repozytorium zaraz po jego utworzeniu. 


Jeśli w repozytorium pojawiły się już jakieś zmiany zanim do niego wstawiono \texttt{.gitignore}, to wtedy należy wykonać ktokroki opisane na stronie: \url{https://stackoverflow.com/questions/38450276/force-git-to-update-gitignore/38451183}

\begin{lstlisting}[basicstyle=\small\ttfamily]
> > > > You will have to clear the existing git cache first.
    git rm -r --cached .
> > > > Once you clear the existing cache, adds/stages all of the files in the current directory and commit
    git add .
    git commit -m "Suitable Message"
> > > > 
\end{lstlisting}


Zalecany schemat współpracy dyplomanta z promotorem polega na wykonywaniu w kolejnych iteracjach następujących kroków:
\begin{itemize}
\item dyplomant edytuje wybraną część pracy, a po skończeniu edycji wrzuca zmiany do zdalnego repozytorium.
\item dyplomant informuje promotora o zakończeniu etapu prac
\item dyplomant może zacząć edycję kolejnego fragmentu pracy, a w tym czasie promotor może dokonać oceny/korekty zmian pobranych ze zdalnego repozytorium
\item promotor wrzuca dokonane przez siebie zmiany do zdalnego repozytorium, informując o tym dyplomanta\
\end{itemize}
Główna zasada tego schematu polega na niedoprowadzaniu do konfliktów (nadpisywanie się zmian). Jeśli jednak takie konflikty nastąpią, można je niwelować poprzez odpowiednie merdżowanie. 

Do wzajemnego informowania można wykorzystać pocztę elektroniczną. Można też spróbować wdrożyć mechanizm zatwierdzania zmian (ang.~\emph{merge requests}). Można też umówić się na sprawdzanie zawartości repozytorium zgodnie z jakimś przyjętym harmonogramem. Ważne, by wiadomo było obu stronom, na jakim schemacie współpracy mają bazować.

Dobrą praktyką jest też wstawianie w kod komentarzy. Przyjętą powszechnie konwencją jest rozpoczynanie komentarzy od:
\begin{itemize}
\item \verb|% TO DO: tekst zalecenia| 
-- jeśli jest to jakieś zalecenie promotora, czy też 
\item \verb|% DONE: tekst wyjaśnienia| 
-- jeśli jakieś zalecenie zostało wykonane przez dyplomanta. 
\end{itemize}

Jako zdalne repozytorium można wykorzystać: github, bitbucket, gitlab (są to serwisy, które pozwalają zarządzać repozytoriami git).
Dobrą praktyką jest też uruchomienie klientów git oferujących graficzny interfejs (jak SourceTree, GitCracken itp.). W narzędziach tych można zobaczyć natychmiast na czym polegały wprowadzone w repozytorium zmiany. 
% !TeX root = ./Dyplom.tex

\chapter{Generowanie reprezentacji wektorowej wypowiedzi}
	Reprezentacje wektorowe słów (ang. word embedding)
\section{BERT}



  Podział na zdania - trza bo:
  \begin{figure}[ht]
    \begin{minipage}{.75\textwidth}\label{fig:word_count}
      \includegraphics[width=\textwidth]{rys03/word_count.png}
    \end{minipage}%
    \begin{minipage}{.25\textwidth}\label{tab:word_count}
      \small
      \begin{tabularx}{\textwidth}{l|l}
        Min & 21 \\ 
        Q1 & 119 \\ 
        Mediana & 219 \\
        Q3 & 494 \\ 
        Górny wąs & 1056 \\
        Max & 28.546 \\
      \end{tabularx}
    \end{minipage}
    \caption{Dystrybucja długości wypowiedzi (zakres do 2000 słów)}
  \end{figure}
% !TeX root = ./Dyplom.tex

\chapter{Analiza tematyczna}
	Analiza tematyczna (ang.\ \emph{topic modeling}) służy do organizowania, wyjaśniania, analizy w czasie, przeszukiwania i streszczania dużych zbiorów danych.
	Najpopularniejszą metodą modelowania tematów jest LDA (ang.\ \emph{Latent Dirichlet Allocation})\cite{LDA}.
	Jest to generatywny model probabilistyczny.
	Dokonuje dyskretyzacji ciągłej przestrzeni tematów na z góry określoną liczbę \emph{t} tematów
		i modeluje dokumenty jako kombinację tematów, dla każdego określając prawdopodobieństwo.
	Każdy temat jest rozkładem prawdopodobieństwa słów.
	Często największe prawdopodobieństwo wystąpienia mają powszechne słowa nie niosące istotnej informacji, np.\ zaimki czy łączniki.
	Wymagana jest lista nieinformatywnych słów (ang.\ \emph{stop words}), które są odfiltrowywane z dokumentów.
	Często lista ta musi być dopasowana do danego zbioru tekstów.
	Dokumenty reprezentowane są jako zbiór słów, w którym zliczane jest wystąpienie każdego słowa --- BOW (ang.\ \emph{bag-of-words}).
	Taki format nie zachowuje informacji o kolejności słów, ani ich semantyce.
	Celem LDA jest znalezienie takich tematów, które mogą posłużyć do odtworzenia oryginalnych rozkładów słów z dokumentów z minimalnym błędem.
	Model nie rozróżnia pomiędzy słowami informatywnymi, a nieinformatywnymi, ma za zadanie jedynie jak najlepiej odzwierciedlać oryginalne dokumenty.
	Z tego powodu najbardziej prawdopodobne słowa w temacie niekoniecznie odzwierciedlają faktyczne tematy dokumentów.
	\todo{LDA dla pl\cite{BoW_PL}}

	Modele oparte o transformery osiągają najlepsze wyniki w wielu zagadnieniach z dziedziny NLP,
		więc powstały również bazujące na nich metody analizy tematycznej.
	Jedną z nich jest BERTopic\cite{BERTopic}, algorytm generujący tematy na podstawie wektorów BERT\@.
	Wykorzystuje UMAP\cite{UMAP} w celu redukcji wymiarowości wektorów do 5 wymiarów.
	Następnie za pomocą algorytmu HDBSCAN\cite{HDBSCAN} wektory grupowane są w klastry.
	Każda grupa reprezentuje pewien temat.
	Reprezentacje tematów generowane są za pomocą wariantu TF-IDF\@,
		gdzie wszystkie dokumenty w klastrze traktowane są jako jeden dokument i porównywane między sobą.
	Poszczególne etapy rozwinięte zostały w kolejnych podrozdziałach.

\section{Redukcja wymiarów}
	Wektory powstałe w wyniku analizy wypowiedzi za pomocą SBERT mają 768 wymiarów.
	W przypadku wykorzystania USE jest to 512 wymiarów, natomiast wektory TF-IDF mają tyle wymiarów,
		ile jest wszystkich tokenów w korpusie (dla wykorzystywanego korpusu jest to ponad 2mln).
	Wykorzystywany algorytm grupowania HDBSCAN działa lepiej na danych w mniejszym wymiarze,
		autorzy testowali algorytm do 50 wymiarów\cite{HDBSCAN}.
	Z tego powodu konieczna jest redukcja wymiaru danych.
	
	Wykorzystano w tym celu algorytm UMAP (ang.\ \emph{Uniform Manifold Approximation and Projection})\cite{UMAP}.
	Oparty jest on o techniki topologicznej analizy danych bazując na geometrii riemannowskiej.
	Konstruuje rozmytą reprezentację topologiczną wysokowymiarowych danych,
		a następnie optymalizuje reprezentację tych danych w docelowym wymiarze tak,
		aby jej rozmyta reprezentacja topologiczna była jak najbardziej podobna mierząc entropią krzyżową.
	W ten sposób algorytm dąży do tego, aby dane w niskowymiarowej przestrzeni jak najlepiej odzwierciedlały topologiczną strukturę oryginalnych danych.
	Dzięki temu zachowane są zarówno lokalne, jak i globalne zależności między danymi.

	Redukcja do zbyt niskiego wymiaru może skutkować zbyt wielkiej utracie informacji,
		podczas gdy redukcja do większego wymiaru może dawać gorsze rezultaty podczas grupowania.
	Postanowiono redukować dane do pięciu wymiarów, tak jak w algorytmie BERTopic.
	Podczas redukcji wymiarowości wektorów SBERT i USE korzystano z metryki kosinusowej.
	Dla wektorów TF-IDF zastosowano metrykę Hellingera, która mierzy podobieństwo rozkładów prawdopodobieństwa
		(wektory tf-idf można traktować jak prawdopodobieństwo znalezienia danego tokenu w dokumencie).
	
	Istotnym parametrem algorytmu UMAP jest \verb|n_neighbors|.
	Balansuje on między skupieniem się na lokalnej strukturze danych, a globalną strukturą.
	Ogranicza liczbę sąsiadujących wektorów, które są analizowane podczas nauki struktury rozmaitości topologicznej danych.
	Dla niskich wartości koncentruje się na lokalnej strukturze nie biorąc pod uwagę pełnego obrazu danych.
	Wysokie wartości skutkują utratą szczegółów, ukazując bardziej ogólne zależności.
	Wpływ tego parametru zobrazowany został na rysunku~\ref{fig:umap} dla wartości 5, 15 oraz 100.
	W celu wizualizacji dane zredukowane zostały do dwóch wymiarów, a kolory odpowiadają klastrom wykrytym przez HDBSCAN.

	\begin{figure}[htb]
		\centering
		\begin{minipage}{.33\textwidth}
			a)\par\medskip % chktex 10
			\missingfigure[figwidth=\linewidth]{UMAP dla n=5}
			n\_neighbors = 5
			% \includegraphics[width=\linewidth]{}
		\end{minipage}%
		\begin{minipage}{.33\textwidth}
			b)\par\medskip % chktex 10
			\missingfigure[figwidth=\linewidth]{UMAP dla n=15}
			n\_neighbors = 15
			% \includegraphics[width=\linewidth]{}
		\end{minipage}%
		\begin{minipage}{.33\textwidth}
			c)\par\medskip % chktex 10
			\missingfigure[figwidth=\linewidth]{UMAP dla n=100}
			n\_neighbors = 100
			% \includegraphics[width=\linewidth]{}
		\end{minipage}
		\caption{Dwuwymiarowa reprezentacja danych przez UMAP w zależności od wartości parametru n\_neighbors}\label{fig:umap} % chktex 9 chktex 10
	\end{figure}
	

\section{Grupowanie}


% !TeX root = ./Dyplom.tex

\chapter{Analiza skuteczności modeli}
\todo{Opisać\cite{Eval_Topics}\cite{U_MASS}\cite{C_NPMI}}

\section{Ewaluacja spójności}
	Ze względu na specyfikę korpusu możliwe jest również sformułowanie metryki wykorzystującej porządek obrad sejmu.
	Każde posiedzenie przebiega zgodnie z ustaloną strukturą, gdzie marszałek zarządza obrady na temat danej ustawy, poprawki czy interpelacji,
		a następnie kolejni posłowie wyrażają swoją opinię na ten temat.
	Można więc założyć, że w obrębie debaty nad jednym punktem obrad, wypowiedzi powinny zostać zakwalifikowane do tego samego zbioru tematycznego.
	Stworzono więc algorytm wykorzystujący tę zależność, przedstawiony na listingu\ref{lst:consistency}.
	Założono, że w zdecydowanej większości przypadków, kolejne dwie wypowiedzi powinny mieć przypisany ten sam temat.

	Algorytm analizuje osobno każdy dzień posiedzeń.
	Wypowiedzi sortowane są zgodnie z kolejnością wynikającą z transkrypcji przemówień.
	Jeśli w ciągu dnia jest mniej niż dwie wypowiedzi, to dzień jest pomijany.
	W zbiorze danych występuje jeden taki przypadek, natomiast mediana wynosi 138.
	Wśród wypowiedzi danego dnia porównywane są tematy par sąsiadujących wypowiedzi.
	Zliczane są zarówno znalezione pasujące pary, jak i różniące się.
	Jeśli temat analizowanej wypowiedzi jest nieznany (algorytm HDBSCAN nie przypisał dokumentu do żadnej grupy),
		to jest ona pomijana.
	Mechanizm ten zapobiega sytuacji, w której duża liczba nieprzypisanych dokumentów zawyżałaby wynik.
	Dzieje się tak ze względu na to, że nieprzypisane dokumenty często stanowią znaczną część zbioru (30--50\%),
		przez co istnieje duże prawdopodobieństwo znalezienia par takich dokumentów.
	Dodatkowo od sumy różniących się par odejmowana jest liczba o 1 mniejsza od liczby tematów danego dnia.
	Działanie to rekompensuje przejścia między tematami, które muszą wystąpić i nie są błędem.
	Takich przejść w ciągu dnia wystąpi co najmniej tyle, ile jest tematów minus jeden.
	Końcowy wynik obliczany jest jako stosunek zgadzających się par do wszystkich zliczonych par.
	
	\begin{lstlisting}[style=algorithm,label=lst:consistency,caption=Algorytm obliczania spójności tematów]
input: speeches sorted by original order of speaking with date and assigned topic identifier
output: consistency score between 0 and 1 with higher score indicating more consistent topics
	
same_per_day := []
different_per_day := []

for each speeches in data grouped by date do
	if len(speeches) < 2
		continue
	
	same := 0
	different := 0
	for i in range(len(speeches) - 1) do
		if speeches[i].topic is unknown
			continue

		if speeches[i].topic = speeches[i + 1].topic
			same := same + 1
		else
			different := different + 1

	topic_count := len(speeches.topic.unique())
	different := different - (topic_count - 1)
	
	same_per_day.append(same)
	different_per_day.append(different)

score = sum(same_per_day) / sum(same_per_day, different_per_day)
return score
	\end{lstlisting}
% !TeX root = ./Dyplom.tex

\chapter{Omówienie wyników}
	

% !TeX root = ./Dyplom.tex

\chapter{Podsumowanie}
	W pracy wykonane zostało kompleksowe porównanie metod kodujących duże i różnorodne dokumenty w wektory zachowując cechy semantyczne tekstu.
	Jako zadanie służące do porównania efektywności modeli językowych wybrano modelowanie tematyczne, które jest problemem uczenia nienadzorowanego.
	Skupiono się na najnowszych algorytmach z tej dziedziny, jednak porównano je również z wiodącą metodą LDA\@.
	Z powodzeniem utworzono modele przewyższające jakością LDA, choć nie wszystkie pierwotne założenia okazały się słuszne.
	Jak pokazała analiza wyników, dokładność zaawansowanych modeli takich jak \emph{Sentence-Transformers} czy \emph{Universal Sentence Encoder}
		maleje dla długich wypowiedzi.
	Okazuje się, że analizując tak długie teksty, lepiej jest skorzystać ze zwykłych metod statystycznych poprzedzonych lematyzacją danych wejściowych.
	Wektory wygenerowane w taki sposób efektywniej wykorzystują duży rozmiar dokumentów,
		podczas gdy pozostałe modele osiągają gorszą dokładność wynikającą z uśredniania wektorów zdań bądź analizy tylko początkowych fragmentów wypowiedzi.

	Wysoce prawdopodobne, że w kontekście innych problemów dokładność wektorów wygenerowanych przez zaawansowane modele językowe dadzą lepsze rezultaty.
	Takim problemem może być przykładowo analiza wydźwięku (ang.\ \emph{sentiment analysis}), czy asymetryczne wyszukiwanie (porównywanie krótkiego zapytania z długimi wypowiedziami).
	Interesującym rozszerzeniem pracy byłoby połączenie modelu tematycznego z analizą wydźwięku.
	Umożliwiłoby to analizę, jakie ugrupowania polityczne wypowiadają się pozytywnie lub negatywnie na dane tematy.
	Nie jest jednak jasne, w jaki sposób dokonać automatycznej ewaluacji dokładności takich modeli.


	\section{Możliwości dalszego usprawnienia modeli}
		W projekcie skorzystano z dostępnych wytrenowanych modeli wielojęzykowych.
		Bardzo możliwe, że wytrenowanie modelu dedykowanego do języka polskiego przyniesie lepsze wyniki.
		Stworzenie zbioru danych treningowych do bezpośredniego wytrenowania modelu \emph{SBERT} nie wydaje się łatwo wykonalne,
			aczkolwiek możliwe jest przekazanie wiedzy z modelu angielskiego na polski.
		Wtedy można by wykorzystać najlepszy polski model wielojęzykowy wg.~rankingu KLEJ\footnote{\url{https://klejbenchmark.com/leaderboard} (dostęp dnia 20 czerwca 2021)} (\emph{XLM-RoBERTa (large) + NKJP}).
		Jest to jednak kosztowne działanie ze względu na rozmiar modelu.
		Obiecującym podejściem jest też analiza dokumentów modelami typu Longformer\cite{Longformer}, które dostosowują modele BERT do znacznie dłuższych sekwencji.
		Nie istnieje jeszcze model w pełni wytrenowany dla języka polskiego, są jednak prowadzone nad tym prace\footnote{\url{https://huggingface.co/clarin-pl/long-former-polish} (dostęp dnia 20 czerwca 2021)}. 



%\bibliographystyle{plalpha}
\bibliographystyle{plabbrv}

%UWAGA: bibliotekę referencji należy przygotować samemu. Dobrym do tego narzędziem jest JabRef.
%       Nazwę przygotowanej biblioteki wpisuje się poniżej bez rozszerzenia 
%       (w tym przypadku jest to "dokumentacja.bib")
\setlength{\bibitemsep}{2pt} % - by zacieśnić wykaz literatury
\bibliography{dokumentacja}
\appendix
\include{dodatekA}
\chapter{Opis załączonej płyty CD/DVD}
Plik Dyplom.pdf zawiera dokument z niniejszą pracą dyplomową.
Dodatkowo, na płycie zamieszczono również dwa foldery:
\begin{description}
	\item[Praca magisterska] zawiera kod źródłowy w języku LaTeX, z którego wygenerowano ten dokument,
	\item[Kod] zawiera kod źródłowy wykorzystany do wykonania badań.
\end{description}

% Poniższe jest niepotrzebne, jeśli nie generuje się indeksu
%%\chapterstyle{noNumbered}
%%\phantomsection % sets an anchor
%%\addcontentsline{toc}{chapter}{Indeks rzeczowy}
%%\printindex

\end{document}
